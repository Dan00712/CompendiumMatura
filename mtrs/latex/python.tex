%% Generated by Sphinx.
\def\sphinxdocclass{jupyterBook}
\documentclass[letterpaper,10pt,english]{jupyterBook}
\ifdefined\pdfpxdimen
   \let\sphinxpxdimen\pdfpxdimen\else\newdimen\sphinxpxdimen
\fi \sphinxpxdimen=.75bp\relax
\ifdefined\pdfimageresolution
    \pdfimageresolution= \numexpr \dimexpr1in\relax/\sphinxpxdimen\relax
\fi
%% let collapsible pdf bookmarks panel have high depth per default
\PassOptionsToPackage{bookmarksdepth=5}{hyperref}
%% turn off hyperref patch of \index as sphinx.xdy xindy module takes care of
%% suitable \hyperpage mark-up, working around hyperref-xindy incompatibility
\PassOptionsToPackage{hyperindex=false}{hyperref}
%% memoir class requires extra handling
\makeatletter\@ifclassloaded{memoir}
{\ifdefined\memhyperindexfalse\memhyperindexfalse\fi}{}\makeatother

\PassOptionsToPackage{warn}{textcomp}

\catcode`^^^^00a0\active\protected\def^^^^00a0{\leavevmode\nobreak\ }
\usepackage{cmap}
\usepackage{fontspec}
\defaultfontfeatures[\rmfamily,\sffamily,\ttfamily]{}
\usepackage{amsmath,amssymb,amstext}
\usepackage{polyglossia}
\setmainlanguage{english}



\setmainfont{FreeSerif}[
  Extension      = .otf,
  UprightFont    = *,
  ItalicFont     = *Italic,
  BoldFont       = *Bold,
  BoldItalicFont = *BoldItalic
]
\setsansfont{FreeSans}[
  Extension      = .otf,
  UprightFont    = *,
  ItalicFont     = *Oblique,
  BoldFont       = *Bold,
  BoldItalicFont = *BoldOblique,
]
\setmonofont{FreeMono}[
  Extension      = .otf,
  UprightFont    = *,
  ItalicFont     = *Oblique,
  BoldFont       = *Bold,
  BoldItalicFont = *BoldOblique,
]



\usepackage[Bjarne]{fncychap}
\usepackage[,numfigreset=1,mathnumfig]{sphinx}

\fvset{fontsize=\small}
\usepackage{geometry}


% Include hyperref last.
\usepackage{hyperref}
% Fix anchor placement for figures with captions.
\usepackage{hypcap}% it must be loaded after hyperref.
% Set up styles of URL: it should be placed after hyperref.
\urlstyle{same}


\usepackage{sphinxmessages}



        % Start of preamble defined in sphinx-jupyterbook-latex %
         \usepackage[Latin,Greek]{ucharclasses}
        \usepackage{unicode-math}
        % fixing title of the toc
        \addto\captionsenglish{\renewcommand{\contentsname}{Contents}}
        \hypersetup{
            pdfencoding=auto,
            psdextra
        }
        % End of preamble defined in sphinx-jupyterbook-latex %
        

\title{MTRS-Compendium}
\date{Apr 22, 2023}
\release{}
\author{Daniel St.}
\newcommand{\sphinxlogo}{\vbox{}}
\renewcommand{\releasename}{}
\makeindex
\begin{document}

\pagestyle{empty}
\sphinxmaketitle
\pagestyle{plain}
\sphinxtableofcontents
\pagestyle{normal}
\phantomsection\label{\detokenize{mtrs::doc}}
\begin{sphinxuseclass}{cell}
\begin{sphinxuseclass}{tag_hide-input}
\end{sphinxuseclass}
\end{sphinxuseclass}
\begin{sphinxuseclass}{cell}
\begin{sphinxuseclass}{tag_hide-input}
\end{sphinxuseclass}
\end{sphinxuseclass}


\begin{DUlineblock}{0em}
\item[] \sphinxstylestrong{\Large SI\sphinxhyphen{}Einheiten}
\end{DUlineblock}


\begin{savenotes}\sphinxattablestart
\centering
\begin{tabulary}{\linewidth}[t]{|T|T|T|}
\hline
\sphinxstyletheadfamily 
\sphinxAtStartPar
Formelzeichen
&\sphinxstyletheadfamily 
\sphinxAtStartPar
Name
&\sphinxstyletheadfamily 
\sphinxAtStartPar
Einheit
\\
\hline
\sphinxAtStartPar
s
&
\sphinxAtStartPar
Länge
&
\sphinxAtStartPar
m
\\
\hline
\sphinxAtStartPar
t
&
\sphinxAtStartPar
Zeit
&
\sphinxAtStartPar
s
\\
\hline
\sphinxAtStartPar
K
&
\sphinxAtStartPar
Temperatur
&
\sphinxAtStartPar
T
\\
\hline
\sphinxAtStartPar
I
&
\sphinxAtStartPar
Stromstärke
&
\sphinxAtStartPar
A
\\
\hline
\sphinxAtStartPar
m
&
\sphinxAtStartPar
Masse
&
\sphinxAtStartPar
kg
\\
\hline
\sphinxAtStartPar

&
\sphinxAtStartPar
Stoffmenge
&
\sphinxAtStartPar
mol
\\
\hline
\sphinxAtStartPar

&
\sphinxAtStartPar
Lichtstärke
&
\sphinxAtStartPar
cd
\\
\hline
\end{tabulary}
\par
\sphinxattableend\end{savenotes}

\begin{DUlineblock}{0em}
\item[] \sphinxstylestrong{\large Einheiten Gleichungen}
\end{DUlineblock}

\sphinxAtStartPar
Come on

\sphinxAtStartPar
Einheiten sollten am ende die richtige Einheit haben
und man kann oft auch oft einfach Werte so zusammenwürfeln damit die Richtige Einheit rauskommt.

\begin{DUlineblock}{0em}
\item[] \sphinxstylestrong{\large Geräteklassen}
\end{DUlineblock}

\sphinxAtStartPar
Genauigkeitsklasse gibt die Genauigkeit von Messgeräten

\sphinxAtStartPar
Feinmessgeräten: unter \(.5\%\)
Betriebsmessgeräte: über \(5\%\)

\begin{DUlineblock}{0em}
\item[] \sphinxstylestrong{\large relative\sphinxhyphen{} und absoluter Fehler}
\end{DUlineblock}

\sphinxAtStartPar
Relativer Fehler:\\
Wert \(\pm\) Fehler in Prozent

\sphinxAtStartPar
Absoluter Fehler:\\
Wert \(\pm\) Fehler in Einheit von Wert

\begin{DUlineblock}{0em}
\item[] \sphinxstylestrong{\large Maßnahmen zur Fehlervermeidung}
\end{DUlineblock}
\begin{itemize}
\item {} 
\sphinxAtStartPar
Korrekte Verwendung

\item {} 
\sphinxAtStartPar
elektromagnetische Fehler vermeiden

\item {} 
\sphinxAtStartPar
richtige Messgeräte verwenden

\item {} 
\sphinxAtStartPar
Messgeräte Nullstellung kontrollieren

\item {} 
\sphinxAtStartPar
Messgeräte Regelmäßig eichen

\item {} 
\sphinxAtStartPar
im richtigen Temperaturbereich verwenden

\item {} 
\sphinxAtStartPar
Bei Analogen Messgeräten im oberen drittel des Bereichs Messen

\end{itemize}

\begin{DUlineblock}{0em}
\item[] \sphinxstylestrong{\large Übersicht über die Messfehler}
\end{DUlineblock}

\begin{DUlineblock}{0em}
\item[] \sphinxstylestrong{\large Systematische Fehler}
\end{DUlineblock}

\sphinxAtStartPar
Der Fehler wird vom systematischen schaltungstechnischen Aufbau bestimmt,
Durch die Analyse der Ursache des Fehlers,
ist der Fehler in Größe und Form bestimmbar

\begin{DUlineblock}{0em}
\item[] \sphinxstylestrong{\large Dynamische Fehler}
\end{DUlineblock}

\sphinxAtStartPar
Nach dem Anlegen der Spannung entstehen Schwingungen
Diese Schwingungen brauchen Zeit zum Ausschwingen,
erreichen aber nie \(0\).
Mann muss warten bis die Schwingung klein genug sind.
Hierzu legt man ein Fehlerband, von z.B. \(3\%\),
sobald die Schwingungen in diesem Band sind kann man Messungen durchführen.

\sphinxAtStartPar
Die Zeit zum Einschwingen nennt man “settling time”

\begin{DUlineblock}{0em}
\item[] \sphinxstylestrong{\large Parallaxe Fehler}
\end{DUlineblock}

\sphinxAtStartPar
Passiert beim Ablesen von analogen Messgeräten,\\
durch Abstand zwischen Zeiger und Messskala

\begin{DUlineblock}{0em}
\item[] \sphinxstylestrong{\large Fehlerfortpflanzung}
\end{DUlineblock}

\begin{DUlineblock}{0em}
\item[] \sphinxstylestrong{\large Addition}
\end{DUlineblock}

\sphinxAtStartPar
\(Z = X + Y = X_m + \Delta X + Y + \Delta Y = \left(X_m + Y_m\right) + \left(\Delta X + \Delta Y\right)\)\\
\(Z = Z_m + \Delta Z\)

\begin{DUlineblock}{0em}
\item[] \sphinxstylestrong{\large Subtraktion}
\end{DUlineblock}

\sphinxAtStartPar
\(Z = X - Y = X_m + \Delta X - \left(Y + \Delta Y\right) = \left(X_m - Y_m\right) + \left(\Delta X - \Delta Y\right)\)\\
\(Z = Z_m + \Delta Z\)

\begin{DUlineblock}{0em}
\item[] \sphinxstylestrong{\large Multiplikation}
\end{DUlineblock}

\sphinxAtStartPar
\(Z = X\cdot Y = \left(X_m + \Delta X\right) \cdot \left(Y_m + \Delta Y\right)\)\\
\(=\left(X_m \cdot Y_m\right) + \left(Y_m \Delta X + X \cdot \Delta Y\right) + \left(\Delta X \cdot \Delta Y\right)\)\\
\(=\left(X_m \cdot Y_m\right) + \left(Y_m \Delta X + X \cdot \Delta Y\right) + \sout{\left(\Delta X \cdot \Delta Y\right)}\)\\
\(=\left(X_m \cdot Y_m\right) + \left(Y_m \Delta X + X \cdot \Delta Y\right) = Z_m + \Delta Z\)

\begin{DUlineblock}{0em}
\item[] \sphinxstylestrong{\large AD/DA}
\end{DUlineblock}

\begin{DUlineblock}{0em}
\item[] \sphinxstylestrong{\large Quantisierungsfehler}
\end{DUlineblock}

\sphinxAtStartPar
Digitale Werte haben begrenzte Genauigkeit.\\
Zum AD\sphinxhyphen{}Wandeln muss auf einen diskreten Wert gerundet werden,
dabei geht Genauigkeit verloren

\begin{DUlineblock}{0em}
\item[] \sphinxstylestrong{\large Umsetzer\sphinxhyphen{}Kennlinien}
\end{DUlineblock}

\begin{DUlineblock}{0em}
\item[] \sphinxstylestrong{\Large Messverfahren}
\end{DUlineblock}

\begin{DUlineblock}{0em}
\item[] \sphinxstylestrong{\large Analoges und Digitales Oszilloskop}
\end{DUlineblock}

\begin{DUlineblock}{0em}
\item[] \sphinxstylestrong{\large Leistungsmessung im 1\sphinxhyphen{}Phasen\sphinxhyphen{}System}
\end{DUlineblock}

\sphinxAtStartPar
Leistung \(= U\cdot I\)

\sphinxAtStartPar
Bei verschiedenen Signalformen gilt immer: \(P = U_{eff}\cdot I_{eff}\)\\
Nur je nach Signalform berechnen sich \(U_{eff}\) und \(I_{eff}\).
Bei sinus:
\begin{itemize}
\item {} 
\sphinxAtStartPar
\(U_{eff} = \frac{\hat{u}}{\sqrt{2}}\)

\item {} 
\sphinxAtStartPar
\(I_{eff} = \frac{\hat{i}}{\sqrt{2}}\)

\end{itemize}

\begin{DUlineblock}{0em}
\item[] \sphinxstylestrong{\large Strom\sphinxhyphen{}Richtige Messung}
\end{DUlineblock}

\begin{sphinxuseclass}{cell}
\begin{sphinxuseclass}{tag_hide-input}\begin{sphinxVerbatimOutput}

\begin{sphinxuseclass}{cell_output}
\noindent\sphinxincludegraphics{{aea635f70fe5b46686e29483b3cb38820b4093fb602d0969a72febe6eab41f2d}.png}

\end{sphinxuseclass}\end{sphinxVerbatimOutput}

\end{sphinxuseclass}
\end{sphinxuseclass}
\begin{sphinxuseclass}{cell}
\begin{sphinxuseclass}{tag_hide-input}\begin{sphinxVerbatimOutput}

\begin{sphinxuseclass}{cell_output}
\noindent\sphinxincludegraphics{{75c0174e718687960c98f709f5702f083b79f8bbbc0cd2a25f42116659b5a67f}.png}

\end{sphinxuseclass}\end{sphinxVerbatimOutput}

\end{sphinxuseclass}
\end{sphinxuseclass}
\sphinxAtStartPar
\(P_r = U\cdot I\) … richtige Leistung

\sphinxAtStartPar
\(P_a = U'\cdot I\) … angezeigte Leistung

\sphinxAtStartPar
\(P_a = (U_{st} + U)\cdot I = P_{st} + P_r\)
\begin{itemize}
\item {} 
\sphinxAtStartPar
\(U'\) … Eingangsspannung

\item {} 
\sphinxAtStartPar
\(U_{st}\)…Spannung am Leistungsmessgerät

\item {} 
\sphinxAtStartPar
\(U\)… Spannung am Widerstand

\end{itemize}

\sphinxAtStartPar
\(P = U\cdot I + I^2\cdot R_A\)

\sphinxAtStartPar
\(R_A\)…Innenwiederstand des Amperemeters

\begin{DUlineblock}{0em}
\item[] \sphinxstylestrong{\large Spannungs\sphinxhyphen{}Richtig Messung}
\end{DUlineblock}

\begin{sphinxuseclass}{cell}
\begin{sphinxuseclass}{tag_hide-input}\begin{sphinxVerbatimOutput}

\begin{sphinxuseclass}{cell_output}
\noindent\sphinxincludegraphics{{580ca7656df0fe79aec277cad299a4fb215112bef92693ce51b60014baaadb47}.png}

\end{sphinxuseclass}\end{sphinxVerbatimOutput}

\end{sphinxuseclass}
\end{sphinxuseclass}
\begin{sphinxuseclass}{cell}
\begin{sphinxuseclass}{tag_hide-input}\begin{sphinxVerbatimOutput}

\begin{sphinxuseclass}{cell_output}
\noindent\sphinxincludegraphics{{c7e6366433bbe59b291e48b20160269131ed3dd2c6d5930bda0749e3baf675f7}.png}

\end{sphinxuseclass}\end{sphinxVerbatimOutput}

\end{sphinxuseclass}
\end{sphinxuseclass}
\sphinxAtStartPar
\(I\cdot U\)  …richtige Leistung

\sphinxAtStartPar
\(I'\cdot U\)…angezeigte Leistung

\sphinxAtStartPar
\(P_a = U\cdot(I_{sp} + I) = P_{sp} + P_r\)
\begin{itemize}
\item {} 
\sphinxAtStartPar
\(I'\) … Eingangsstrom

\item {} 
\sphinxAtStartPar
\(I_{st}\)…Strom durch Leistungsmessgerät

\item {} 
\sphinxAtStartPar
\(I\)… Strom am Widerstand

\end{itemize}

\sphinxAtStartPar
\(P = U\cdot I - \frac{U^2}{R_V}\)

\sphinxAtStartPar
\(R_V\)…Innenwiederstand Voltmeter

\begin{sphinxuseclass}{cell}
\begin{sphinxuseclass}{tag_hide-input}\begin{sphinxVerbatimOutput}

\begin{sphinxuseclass}{cell_output}
\noindent\sphinxincludegraphics{{9cc6185ccc0af4cf8dcbdf510cec19913a3055a981a88c1d29c8c78cb2ff481a}.png}

\end{sphinxuseclass}\end{sphinxVerbatimOutput}

\end{sphinxuseclass}
\end{sphinxuseclass}
\begin{DUlineblock}{0em}
\item[] \sphinxstylestrong{\large Leistungsarten}
\end{DUlineblock}

\sphinxAtStartPar
\(P = U\cdot I\cdot (cos\varphi + i\cdot sin\varphi)\)

\sphinxAtStartPar
\(\varphi\) ist die Phasenverschiebung zwischen Spannung und Strom
\(\varphi = \varphi_u - \varphi_i\)

\begin{DUlineblock}{0em}
\item[] \sphinxstylestrong{\large Wirkleistungsmessung}
\end{DUlineblock}

\sphinxAtStartPar
\(P = U\cdot I \cdot cos \varphi\)

\sphinxAtStartPar
Realteil der komplexen Leistung.

\sphinxAtStartPar
\(P = \frac{1}{T}\cdot \int_0^T u(t)\cdot i(t) dt = U_{eff} \cdot I_{eff}\cdot cos\varphi\)
\begin{itemize}
\item {} 
\sphinxAtStartPar
\(u(t) = \hat{u}\cdot sin(\omega\cdot t + \varphi_u\)

\item {} 
\sphinxAtStartPar
\(i(t) = \hat{i}\cdot sin(\omega\cdot t + \varphi_i\)

\item {} 
\sphinxAtStartPar
\(U_{eff} = \frac{\hat{u}}{\sqrt{2}}\)

\item {} 
\sphinxAtStartPar
\(I_{eff} = \frac{\hat{i}}{\sqrt{2}}\)

\end{itemize}

\begin{DUlineblock}{0em}
\item[] \sphinxstylestrong{\large Blindleistunsmessung}
\end{DUlineblock}

\sphinxAtStartPar
\(Q = U\cdot I \cdot cos \varphi\)

\sphinxAtStartPar
Imaginärteil der komplexen Leistung.

\sphinxAtStartPar
\(Q = U_{eff} \cdot I_{eff}\cdot sin\varphi\)
\begin{itemize}
\item {} 
\sphinxAtStartPar
\(u(t) = \hat{u}\cdot sin(\omega\cdot t + \varphi_u\)

\item {} 
\sphinxAtStartPar
\(i(t) = \hat{i}\cdot sin(\omega\cdot t + \varphi_i\)

\item {} 
\sphinxAtStartPar
\(U_{eff} = \frac{\hat{u}}{\sqrt{2}}\)

\item {} 
\sphinxAtStartPar
\(I_{eff} = \frac{\hat{i}}{\sqrt{2}}\)

\end{itemize}

\sphinxAtStartPar
Wir über \(90°\) (\(\pi~rad\)) Phasenverschiebung gemessen.
Bspw. in Spannungspfad.

\begin{DUlineblock}{0em}
\item[] \sphinxstylestrong{\large Scheinleistungsmessung}
\end{DUlineblock}

\sphinxAtStartPar
Die Komplexe Leistung, zusammengesetzt aus Wirk\sphinxhyphen{} und Blindleistung

\sphinxAtStartPar
\(S = P + j\cdot Q = |S| \cdot cos\varphi + j\cdot |S| \cdot sin\varphi\)

\sphinxAtStartPar
\(S^2 = P^2 + Q^2\)

\begin{DUlineblock}{0em}
\item[] \sphinxstylestrong{\large Sensorik}
\end{DUlineblock}

\begin{DUlineblock}{0em}
\item[] \sphinxstylestrong{\large Passive Sensoren}
\end{DUlineblock}

\sphinxAtStartPar
Messgröße beeinflusst einen elektrischen Zustand bspw. Widerstand oder Kapazität


\begin{savenotes}\sphinxattablestart
\centering
\begin{tabulary}{\linewidth}[t]{|T|T|T|}
\hline
\sphinxstyletheadfamily 
\sphinxAtStartPar
Sensor
&\sphinxstyletheadfamily 
\sphinxAtStartPar
einwirkende Größe
&\sphinxstyletheadfamily 
\sphinxAtStartPar
beeinflusste Größe
\\
\hline
\sphinxAtStartPar

&
\sphinxAtStartPar

&
\sphinxAtStartPar

\\
\hline
\sphinxAtStartPar
Thermometer
&
\sphinxAtStartPar
Temperatur
&
\sphinxAtStartPar
ohmscher Widerstand
\\
\hline
\sphinxAtStartPar
Dehnungsmessstreifen
&
\sphinxAtStartPar
Längenänderung
&
\sphinxAtStartPar
ohmscher Widerstand
\\
\hline
\sphinxAtStartPar
Fotowiderstand
&
\sphinxAtStartPar
Lichtstärke
&
\sphinxAtStartPar
ohmscher Widerstand
\\
\hline
\sphinxAtStartPar

&
\sphinxAtStartPar

&
\sphinxAtStartPar

\\
\hline
\sphinxAtStartPar
Induktive Sensoren
&
\sphinxAtStartPar
Länge Winkel
&
\sphinxAtStartPar
Induktivität
\\
\hline
\sphinxAtStartPar
Kapazitive Sensoren
&
\sphinxAtStartPar
Länge, Winkel
&
\sphinxAtStartPar
Kapazität
\\
\hline
\end{tabulary}
\par
\sphinxattableend\end{savenotes}

\begin{DUlineblock}{0em}
\item[] \sphinxstylestrong{\large Aktive Sensoren}
\end{DUlineblock}

\sphinxAtStartPar
Aktive Sensoren benötigen keine externe Energie und geben ein Signal und gibt Leistung an das Massglied ab.

\sphinxAtStartPar
Oft nur Änderungen erkennbar, bc PHYSICS


\begin{savenotes}\sphinxattablestart
\centering
\begin{tabulary}{\linewidth}[t]{|T|T|T|}
\hline
\sphinxstyletheadfamily 
\sphinxAtStartPar
Sensor
&\sphinxstyletheadfamily 
\sphinxAtStartPar
einwirkende Größe
&\sphinxstyletheadfamily 
\sphinxAtStartPar
beeinflusste Größe
\\
\hline
\sphinxAtStartPar

&
\sphinxAtStartPar

&
\sphinxAtStartPar

\\
\hline
\sphinxAtStartPar
Thermoelement
&
\sphinxAtStartPar
Temperatur
&
\sphinxAtStartPar
Spannung
\\
\hline
\sphinxAtStartPar
Fotoelement
&
\sphinxAtStartPar
Lichtstärke
&
\sphinxAtStartPar
Spannung \& Strom
\\
\hline
\sphinxAtStartPar
Piezokristall
&
\sphinxAtStartPar
Druck
&
\sphinxAtStartPar
Ladung (Spannung)
\\
\hline
\end{tabulary}
\par
\sphinxattableend\end{savenotes}

\begin{DUlineblock}{0em}
\item[] \sphinxstylestrong{\large Temperatursensoren}
\end{DUlineblock}

\sphinxAtStartPar
In Wheatstone\sphinxhyphen{}Messbrücken eingesetzt

\begin{DUlineblock}{0em}
\item[] \sphinxstylestrong{\large Platinsensoren}
\end{DUlineblock}

\sphinxAtStartPar
\sphinxstylestrong{Temperaturbereich:}
\begin{itemize}
\item {} 
\sphinxAtStartPar
\(\theta = [-200°C; 800°C]\)

\item {} 
\sphinxAtStartPar
\(R_\theta = R_0\cdot(1 + \alpha_{Pt}\Delta \theta)\)

\end{itemize}

\sphinxAtStartPar
\(\alpha_{Pt} = 3.9E-3 K^{-1}\)

\sphinxAtStartPar
\(Pt100 \Rightarrow 100 \Omega\)
\(Pt1000 \Rightarrow 1000 \Omega\)

\begin{DUlineblock}{0em}
\item[] \sphinxstylestrong{\large Silizium\sphinxhyphen{}Sensor}
\end{DUlineblock}

\sphinxAtStartPar
VT:
\begin{itemize}
\item {} 
\sphinxAtStartPar
Geringe Kosten

\item {} 
\sphinxAtStartPar
hoher Temperaturkoeffizient

\end{itemize}

\sphinxAtStartPar
NT:
\begin{itemize}
\item {} 
\sphinxAtStartPar
nichtlinear

\item {} 
\sphinxAtStartPar
kleiner Messbereich

\end{itemize}

\sphinxAtStartPar
\(R_{\theta} = R_0\cdot\left(1 + \alpha\Delta\theta + \beta\Delta\theta^2\right)\)

\sphinxAtStartPar
KTY10: \(R_{25} = 2000\Omega,\qquad\alpha_{25} = 7.37E-3 K^{-1}\)

\begin{DUlineblock}{0em}
\item[] \sphinxstylestrong{\large PTC Widerstand}
\end{DUlineblock}

\sphinxAtStartPar
PTC…Positive Temperature Coefficient, Kaltleiter

\sphinxAtStartPar
\sphinxstylestrong{praktischer Bereich:}
\begin{itemize}
\item {} 
\sphinxAtStartPar
\(\alpha = [+7\frac{\%}{C}, +70\frac{\%}{C}]\)

\item {} 
\sphinxAtStartPar
\(\theta = [-20°C; 200°C]\)

\end{itemize}

\begin{DUlineblock}{0em}
\item[] \sphinxstylestrong{\large NTC Widerstand}
\end{DUlineblock}

\sphinxAtStartPar
NTC…Negative Temperature Coefficient

\sphinxAtStartPar
\(\alpha_N = \left[-2\frac{\%}{°C}; -6\frac{\%}{°C}\right]\)

\sphinxAtStartPar
\(R_T = R_N\cdot e^{B\cdot\left( \frac{1}{T} - \frac{1}{T_N} \right)}\)

\sphinxAtStartPar
\(T\)…absolute Temperatur\\
\(B\)…Materialkonstante\\
\(R_N\)…Nennwiderstand bei Nenntemperatur \(T_N\)

\sphinxAtStartPar
Messung wird über Widerstandsmessung gemacht

\begin{DUlineblock}{0em}
\item[] \sphinxstylestrong{\large Thermoelemente}
\end{DUlineblock}

\sphinxAtStartPar
Durch unterschiedliche Dehnung entsteht, durch \sphinxstyleemphasis{Seebeck\sphinxhyphen{}Effekt}, eine Spannung.\\
Spannung ist Temperaturabhängig

\sphinxAtStartPar
VT:
\begin{itemize}
\item {} 
\sphinxAtStartPar
einfach

\item {} 
\sphinxAtStartPar
weiter Temperaturbereich

\item {} 
\sphinxAtStartPar
keine Selbstheizung (aktiver Sensor)

\item {} 
\sphinxAtStartPar
unabhängig von der Drahtgeometrie

\item {} 
\sphinxAtStartPar
Leitungswiederstände spielen kaum eine Rolle

\end{itemize}

\sphinxAtStartPar
NT:
\begin{itemize}
\item {} 
\sphinxAtStartPar
kleine Spannungen

\item {} 
\sphinxAtStartPar
nichtlineare Kennlinie

\item {} 
\sphinxAtStartPar
aufwendige Kompensation

\item {} 
\sphinxAtStartPar
Ausgleichung erforderlich

\item {} 
\sphinxAtStartPar
manche Drähte sind schlecht verarbeitbar

\end{itemize}

\begin{DUlineblock}{0em}
\item[] \sphinxstylestrong{\large Temperatur\sphinxhyphen{}Fixpunkt}
\end{DUlineblock}

\sphinxAtStartPar
zwei gleichartige Thermoelemente gegeneinander in Reihe.\\
Bei gleicher Temperatur heben sich die Thermospannungen auf.

\sphinxAtStartPar
Vergleichsstelle dient Wasser\sphinxhyphen{}Eis Thermospannung hängt dann nur vom Temperaturunterschied zwischen den Thermoelementen ab.

\begin{DUlineblock}{0em}
\item[] \sphinxstylestrong{\large Isothermal\sphinxhyphen{}Block}
\end{DUlineblock}

\sphinxAtStartPar
Thermoelement Drahtpaar und misst die Differenz der Temperatur der Verbindungsstelle der beiden Metalle.

\begin{DUlineblock}{0em}
\item[] \sphinxstylestrong{\large Spannungsabhängiger Widerstand}
\end{DUlineblock}

\sphinxAtStartPar
VDR…Voltage Dependent Resistor

\sphinxAtStartPar
Geräte vor Überspannung zu Schützen\\
VDR zeichnen sich durch starke Spannungsabhängigkeit. VDR schalten im ns\sphinxhyphen{}Bereich

\sphinxAtStartPar
Im Grunde Spannung bleibt fast auf konstanten Wert.

\begin{sphinxuseclass}{cell}
\begin{sphinxuseclass}{tag_hide-input}\begin{sphinxVerbatimOutput}

\begin{sphinxuseclass}{cell_output}
\noindent\sphinxincludegraphics{{0310bd44243859181ee4300f1839c17e9627b61dd6dd2c31f8b9639a1969177c}.png}

\end{sphinxuseclass}\end{sphinxVerbatimOutput}

\end{sphinxuseclass}
\end{sphinxuseclass}
\sphinxAtStartPar
Mit steigender Spannung baut sich elektrisches Feld auf, welches die pn\sphinxhyphen{}Sperrschicht der Elementarzellen teilweise abbaut.

\sphinxAtStartPar
\(U = C\cdot I^{\beta}\)\\
\(C = 15~ ...~ 5000\Omega\)\\
\(\beta = 0.15...0.4 \text{Regelfaktor}\)

\sphinxAtStartPar
Der Regelfaktor zeigt fast keine Temperaturabhängigkeit.

\sphinxAtStartPar
Bei VDR in Serie addieren sich die Spannungsabfälle.
Bei Parallelschaltung können die Widerstände überlastet werden.

\sphinxAtStartPar
\sphinxstylestrong{Anwendungen:}\\
Spannungsbegrenzung ist notwendig wenn hohe Störspannungen auftreten können
\begin{itemize}
\item {} 
\sphinxAtStartPar
SURGE \sphinxhyphen{} Blitzschlag (\(\leq 2kV, \leq 100kHz, I\approx kA\))

\item {} 
\sphinxAtStartPar
BURST \sphinxhyphen{} geschaltete Induktivitäten (\(2...8 kV, 100...200 MHz\))

\item {} 
\sphinxAtStartPar
ESD \sphinxhyphen{} statische Aufladung (\(8~...~25 kV, \leq GHz\))

\end{itemize}

\sphinxAtStartPar
\sphinxstylestrong{Funkenlöschung:}\\
Spannungsspitzen ab \(\approx 300V\) können Funken bilden.\\
Abhilfe VDR parallel zum Kontakt

\begin{DUlineblock}{0em}
\item[] \sphinxstylestrong{\large Fotowiderstand}
\end{DUlineblock}

\sphinxAtStartPar
LDR…Light Dependent Resistor

\sphinxAtStartPar
Fotowiderstand ändert sich mit Beleuchtungsstärke

\sphinxAtStartPar
Beste Form Mäanderform

\sphinxAtStartPar
Geringer Elektronenabstand alle vom Licht freigesetzten Elektronen aufgenommen werden können.

\begin{DUlineblock}{0em}
\item[] \sphinxstylestrong{\large Messbrücke}
\end{DUlineblock}

\sphinxAtStartPar
Allgemein:
\(U_b = U_0 \cdot \left(\frac{R_2}{R_1 + R_2} - \frac{R_4}{R_4+R_3}\right)\)

\begin{DUlineblock}{0em}
\item[] \sphinxstylestrong{\large Viertelbrücke}
\end{DUlineblock}

\sphinxAtStartPar
\sphinxstylestrong{Viertel}brücke \(\Rightarrow\) 1 von 4 Widerständen änderbar

\begin{sphinxuseclass}{cell}
\begin{sphinxuseclass}{tag_hide-input}\begin{sphinxVerbatimOutput}

\begin{sphinxuseclass}{cell_output}
\noindent\sphinxincludegraphics{{087f9c94381c55ebe9c26b1c3b92548f4b99cc2782a859c7758e49f8c4fe57f9}.png}

\end{sphinxuseclass}\end{sphinxVerbatimOutput}

\end{sphinxuseclass}
\end{sphinxuseclass}
\sphinxAtStartPar
\(U_B = U_{R2} - U_{R4} = U_0 \cdot \left( \frac{R_2}{R_1 + R_2} - \frac{R_4}{R_3 + R_4} \right)\)\\
\(R_1 = R_3 = R_4 = R\)\\
\(R_2 = R + \Delta R\)\\
\(\therefore U_B = U_0 \cdot \frac{\Delta R}{4R + \Delta R} \approx U_0 \cdot \frac{\Delta R}{4R}\)

\begin{DUlineblock}{0em}
\item[] \sphinxstylestrong{\large Halbbrücke}
\end{DUlineblock}

\sphinxAtStartPar
\sphinxstylestrong{Halb}brücke \(\Rightarrow\) 2 von 4 Widerständen änderbar

\begin{sphinxuseclass}{cell}
\begin{sphinxuseclass}{tag_hide-input}\begin{sphinxVerbatimOutput}

\begin{sphinxuseclass}{cell_output}
\noindent\sphinxincludegraphics{{9f577c1cc9b56fa61e47684ebfa4830e744d0be766ea98f7ddf5eb73b6952f96}.png}

\end{sphinxuseclass}\end{sphinxVerbatimOutput}

\end{sphinxuseclass}
\end{sphinxuseclass}
\sphinxAtStartPar
\(U_B = U_{R2} - U_{R4} = U_0 \cdot \left( \frac{R_2}{R_1 + R_2} - \frac{R_4}{R_3 + R_4} \right)\)\\
\(R_1 = R_4 = R\)\\
\(R_2 = R_3 = R + \Delta R\)\\
\(\therefore U_B = U_0 \cdot \frac{\Delta R}{2R}\)
Doppelte Spannungsempfindlichkeit im Vergleich zur Viertelbrücke

\begin{DUlineblock}{0em}
\item[] \sphinxstylestrong{\large Vollbrücke}
\end{DUlineblock}

\sphinxAtStartPar
Alle Widerstände sind veränderbar
und in der Form \(R \pm \Delta R\),
wobei \(R_1\) und \(R_2\) sowie \(R_3\) und \(R_4\) gegengleich sind

\sphinxAtStartPar
\(U_B = U_0 \cdot \frac{\Delta R}{R}\)

\sphinxAtStartPar
Wieder Doppelte Empfindlichkeit im Vergleich zur Halbbrücke

\begin{DUlineblock}{0em}
\item[] \sphinxstylestrong{\large Digitale Frequenzmessung}
\end{DUlineblock}

\begin{DUlineblock}{0em}
\item[] \sphinxstylestrong{\large Schaltung}
\end{DUlineblock}

\begin{sphinxuseclass}{cell}\begin{sphinxVerbatimInput}

\begin{sphinxuseclass}{cell_input}
\begin{sphinxVerbatim}[commandchars=\\\{\}]
\PYG{k}{with} \PYG{n}{schemdraw}\PYG{o}{.}\PYG{n}{Drawing}\PYG{p}{(}\PYG{p}{)} \PYG{k}{as} \PYG{n}{d}\PYG{p}{:}
    \PYG{n}{d} \PYG{o}{+}\PYG{o}{=} \PYG{n}{elm}\PYG{o}{.}\PYG{n}{Line}\PYG{p}{(}\PYG{p}{)}\PYG{o}{.}\PYG{n}{idot}\PYG{p}{(}\PYG{n+nb}{open}\PYG{o}{=}\PYG{k+kc}{True}\PYG{p}{)}\PYG{o}{.}\PYG{n}{label}\PYG{p}{(}\PYG{l+s+s1}{\PYGZsq{}}\PYG{l+s+s1}{\PYGZdl{}f\PYGZus{}x\PYGZdl{}}\PYG{l+s+s1}{\PYGZsq{}}\PYG{p}{)}
    \PYG{n}{d} \PYG{o}{+}\PYG{o}{=} \PYG{n}{elm}\PYG{o}{.}\PYG{n}{Ic}\PYG{p}{(}\PYG{n}{pins}\PYG{o}{=}\PYG{p}{[}
        \PYG{n}{elm}\PYG{o}{.}\PYG{n}{IcPin}\PYG{p}{(}\PYG{n}{side}\PYG{o}{=}\PYG{l+s+s1}{\PYGZsq{}}\PYG{l+s+s1}{left}\PYG{l+s+s1}{\PYGZsq{}}\PYG{p}{,} \PYG{n}{anchorname}\PYG{o}{=}\PYG{l+s+s1}{\PYGZsq{}}\PYG{l+s+s1}{inp}\PYG{l+s+s1}{\PYGZsq{}}\PYG{p}{)}\PYG{p}{,}
        \PYG{n}{elm}\PYG{o}{.}\PYG{n}{IcPin}\PYG{p}{(}\PYG{n}{side}\PYG{o}{=}\PYG{l+s+s1}{\PYGZsq{}}\PYG{l+s+s1}{right}\PYG{l+s+s1}{\PYGZsq{}}\PYG{p}{,} \PYG{n}{anchorname}\PYG{o}{=}\PYG{l+s+s1}{\PYGZsq{}}\PYG{l+s+s1}{oup}\PYG{l+s+s1}{\PYGZsq{}}\PYG{p}{)}
    \PYG{p}{]}\PYG{p}{)}\PYG{o}{.}\PYG{n}{anchor}\PYG{p}{(}\PYG{l+s+s1}{\PYGZsq{}}\PYG{l+s+s1}{inp}\PYG{l+s+s1}{\PYGZsq{}}\PYG{p}{)}\PYG{o}{.}\PYG{n}{drop}\PYG{p}{(}\PYG{l+s+s1}{\PYGZsq{}}\PYG{l+s+s1}{oup}\PYG{l+s+s1}{\PYGZsq{}}\PYG{p}{)}\PYG{o}{.}\PYG{n}{label}\PYG{p}{(}\PYG{l+s+s1}{\PYGZsq{}}\PYG{l+s+s1}{Schmitt\PYGZhy{}Trigger}\PYG{l+s+s1}{\PYGZsq{}}\PYG{p}{,} \PYG{l+s+s1}{\PYGZsq{}}\PYG{l+s+s1}{B}\PYG{l+s+s1}{\PYGZsq{}}\PYG{p}{)}\PYG{o}{.}\PYG{n}{label}\PYG{p}{(}\PYG{l+s+s1}{\PYGZsq{}}\PYG{l+s+se}{\PYGZbs{}u238e}\PYG{l+s+s1}{\PYGZsq{}}\PYG{p}{,} \PYG{l+s+s1}{\PYGZsq{}}\PYG{l+s+s1}{center}\PYG{l+s+s1}{\PYGZsq{}}\PYG{p}{)}

    \PYG{n}{d} \PYG{o}{+}\PYG{o}{=} \PYG{n}{elm}\PYG{o}{.}\PYG{n}{Line}\PYG{p}{(}\PYG{p}{)}\PYG{o}{.}\PYG{n}{length}\PYG{p}{(}\PYG{l+m+mi}{6}\PYG{p}{)}
    \PYG{n}{d} \PYG{o}{+}\PYG{o}{=} \PYG{p}{(}\PYG{n}{ACmp} \PYG{o}{:=} \PYG{n}{lgc}\PYG{o}{.}\PYG{n}{And}\PYG{p}{(}\PYG{p}{)}\PYG{o}{.}\PYG{n}{anchor}\PYG{p}{(}\PYG{l+s+s1}{\PYGZsq{}}\PYG{l+s+s1}{in1}\PYG{l+s+s1}{\PYGZsq{}}\PYG{p}{)}\PYG{o}{.}\PYG{n}{label}\PYG{p}{(}\PYG{l+s+s1}{\PYGZsq{}}\PYG{l+s+s1}{AND\PYGZhy{}Gate}\PYG{l+s+s1}{\PYGZsq{}}\PYG{p}{)}\PYG{o}{.}\PYG{n}{drop}\PYG{p}{(}\PYG{l+s+s1}{\PYGZsq{}}\PYG{l+s+s1}{out}\PYG{l+s+s1}{\PYGZsq{}}\PYG{p}{)}\PYG{p}{)}

    \PYG{n}{d} \PYG{o}{+}\PYG{o}{=} \PYG{p}{(}\PYG{n}{counter} \PYG{o}{:=} \PYG{n}{elm}\PYG{o}{.}\PYG{n}{Ic}\PYG{p}{(}\PYG{n}{pins}\PYG{o}{=}\PYG{p}{[}
        \PYG{n}{elm}\PYG{o}{.}\PYG{n}{IcPin}\PYG{p}{(}\PYG{n}{side} \PYG{o}{=} \PYG{l+s+s1}{\PYGZsq{}}\PYG{l+s+s1}{left}\PYG{l+s+s1}{\PYGZsq{}}\PYG{p}{,} \PYG{n}{anchorname}\PYG{o}{=}\PYG{l+s+s1}{\PYGZsq{}}\PYG{l+s+s1}{in1}\PYG{l+s+s1}{\PYGZsq{}}\PYG{p}{)}\PYG{p}{,}
        \PYG{n}{elm}\PYG{o}{.}\PYG{n}{IcPin}\PYG{p}{(}\PYG{n}{side} \PYG{o}{=} \PYG{l+s+s1}{\PYGZsq{}}\PYG{l+s+s1}{bottom}\PYG{l+s+s1}{\PYGZsq{}}\PYG{p}{,} \PYG{n}{anchorname}\PYG{o}{=}\PYG{l+s+s1}{\PYGZsq{}}\PYG{l+s+s1}{out1}\PYG{l+s+s1}{\PYGZsq{}}\PYG{p}{)}\PYG{p}{,}
        \PYG{n}{elm}\PYG{o}{.}\PYG{n}{IcPin}\PYG{p}{(}\PYG{n}{side} \PYG{o}{=} \PYG{l+s+s1}{\PYGZsq{}}\PYG{l+s+s1}{bottom}\PYG{l+s+s1}{\PYGZsq{}}\PYG{p}{,} \PYG{n}{anchorname}\PYG{o}{=}\PYG{l+s+s1}{\PYGZsq{}}\PYG{l+s+s1}{out2}\PYG{l+s+s1}{\PYGZsq{}}\PYG{p}{)}
    \PYG{p}{]}\PYG{p}{,} \PYG{n}{edgepadW}\PYG{o}{=}\PYG{l+m+mi}{2}\PYG{p}{)}\PYG{o}{.}\PYG{n}{anchor}\PYG{p}{(}\PYG{l+s+s1}{\PYGZsq{}}\PYG{l+s+s1}{in1}\PYG{l+s+s1}{\PYGZsq{}}\PYG{p}{)}\PYG{o}{.}\PYG{n}{label}\PYG{p}{(}\PYG{l+s+s1}{\PYGZsq{}}\PYG{l+s+s1}{n\PYGZhy{}Bit Counter}\PYG{l+s+s1}{\PYGZsq{}}\PYG{p}{,} \PYG{l+s+s1}{\PYGZsq{}}\PYG{l+s+s1}{center}\PYG{l+s+s1}{\PYGZsq{}}\PYG{p}{)}\PYG{o}{.}\PYG{n}{drop}\PYG{p}{(}\PYG{l+s+s1}{\PYGZsq{}}\PYG{l+s+s1}{out1}\PYG{l+s+s1}{\PYGZsq{}}\PYG{p}{)}\PYG{p}{)}

    \PYG{n}{d}\PYG{o}{.}\PYG{n}{here} \PYG{o}{=} \PYG{p}{(}\PYG{l+m+mi}{0}\PYG{p}{,} \PYG{o}{\PYGZhy{}}\PYG{l+m+mi}{6}\PYG{p}{)}
    \PYG{n}{d} \PYG{o}{+}\PYG{o}{=} \PYG{n}{elm}\PYG{o}{.}\PYG{n}{Line}\PYG{p}{(}\PYG{p}{)}\PYG{o}{.}\PYG{n}{idot}\PYG{p}{(}\PYG{n+nb}{open}\PYG{o}{=}\PYG{k+kc}{True}\PYG{p}{)}\PYG{o}{.}\PYG{n}{label}\PYG{p}{(}\PYG{l+s+s1}{\PYGZsq{}}\PYG{l+s+s1}{\PYGZdl{}f\PYGZus{}}\PYG{l+s+si}{\PYGZob{}ref\PYGZcb{}}\PYG{l+s+s1}{\PYGZdl{}}\PYG{l+s+s1}{\PYGZsq{}}\PYG{p}{)}
    \PYG{n}{d} \PYG{o}{+}\PYG{o}{=} \PYG{n}{elm}\PYG{o}{.}\PYG{n}{Ic}\PYG{p}{(}\PYG{n}{pins}\PYG{o}{=}\PYG{p}{[}
        \PYG{n}{elm}\PYG{o}{.}\PYG{n}{IcPin}\PYG{p}{(}\PYG{n}{side}\PYG{o}{=}\PYG{l+s+s1}{\PYGZsq{}}\PYG{l+s+s1}{left}\PYG{l+s+s1}{\PYGZsq{}}\PYG{p}{,} \PYG{n}{anchorname}\PYG{o}{=}\PYG{l+s+s1}{\PYGZsq{}}\PYG{l+s+s1}{in1}\PYG{l+s+s1}{\PYGZsq{}}\PYG{p}{)}\PYG{p}{,}
        \PYG{n}{elm}\PYG{o}{.}\PYG{n}{IcPin}\PYG{p}{(}\PYG{n}{side}\PYG{o}{=}\PYG{l+s+s1}{\PYGZsq{}}\PYG{l+s+s1}{right}\PYG{l+s+s1}{\PYGZsq{}}\PYG{p}{,} \PYG{n}{anchorname}\PYG{o}{=}\PYG{l+s+s1}{\PYGZsq{}}\PYG{l+s+s1}{out}\PYG{l+s+s1}{\PYGZsq{}}\PYG{p}{)}
    \PYG{p}{]}\PYG{p}{,} \PYG{n}{edgepadW}\PYG{o}{=}\PYG{l+m+mi}{2}\PYG{p}{)}\PYG{o}{.}\PYG{n}{label}\PYG{p}{(}\PYG{l+s+s1}{\PYGZsq{}}\PYG{l+s+s1}{Vorteiler}\PYG{l+s+se}{\PYGZbs{}n}\PYG{l+s+s1}{\PYGZdl{}N\PYGZus{}}\PYG{l+s+si}{\PYGZob{}ref\PYGZcb{}}\PYG{l+s+s1}{\PYGZdl{}}\PYG{l+s+s1}{\PYGZsq{}}\PYG{p}{)}\PYG{o}{.}\PYG{n}{anchor}\PYG{p}{(}\PYG{l+s+s1}{\PYGZsq{}}\PYG{l+s+s1}{in1}\PYG{l+s+s1}{\PYGZsq{}}\PYG{p}{)}\PYG{o}{.}\PYG{n}{drop}\PYG{p}{(}\PYG{l+s+s1}{\PYGZsq{}}\PYG{l+s+s1}{out}\PYG{l+s+s1}{\PYGZsq{}}\PYG{p}{)}

    \PYG{k}{def} \PYG{n+nf}{t\PYGZus{}ff}\PYG{p}{(}\PYG{p}{)}\PYG{p}{:}
        \PYG{k}{return} \PYG{n}{elm}\PYG{o}{.}\PYG{n}{Ic}\PYG{p}{(}\PYG{n}{pins}\PYG{o}{=}\PYG{p}{[}
            \PYG{n}{elm}\PYG{o}{.}\PYG{n}{IcPin}\PYG{p}{(}\PYG{l+s+s1}{\PYGZsq{}}\PYG{l+s+s1}{\PYGZgt{}}\PYG{l+s+s1}{\PYGZsq{}}\PYG{p}{,} \PYG{n}{side}\PYG{o}{=}\PYG{l+s+s1}{\PYGZsq{}}\PYG{l+s+s1}{left}\PYG{l+s+s1}{\PYGZsq{}}\PYG{p}{)}\PYG{p}{,}
            \PYG{n}{elm}\PYG{o}{.}\PYG{n}{IcPin}\PYG{p}{(}\PYG{l+s+s1}{\PYGZsq{}}\PYG{l+s+s1}{T}\PYG{l+s+s1}{\PYGZsq{}}\PYG{p}{,} \PYG{n}{side}\PYG{o}{=}\PYG{l+s+s1}{\PYGZsq{}}\PYG{l+s+s1}{left}\PYG{l+s+s1}{\PYGZsq{}}\PYG{p}{)}\PYG{p}{,}
            \PYG{n}{elm}\PYG{o}{.}\PYG{n}{IcPin}\PYG{p}{(}\PYG{l+s+s1}{\PYGZsq{}}\PYG{l+s+s1}{\PYGZdl{}}\PYG{l+s+s1}{\PYGZbs{}}\PYG{l+s+s1}{overline}\PYG{l+s+si}{\PYGZob{}Q\PYGZcb{}}\PYG{l+s+s1}{\PYGZdl{}}\PYG{l+s+s1}{\PYGZsq{}}\PYG{p}{,} \PYG{n}{side}\PYG{o}{=}\PYG{l+s+s1}{\PYGZsq{}}\PYG{l+s+s1}{right}\PYG{l+s+s1}{\PYGZsq{}}\PYG{p}{,} \PYG{n}{anchorname}\PYG{o}{=}\PYG{l+s+s1}{\PYGZsq{}}\PYG{l+s+s1}{nQ}\PYG{l+s+s1}{\PYGZsq{}}\PYG{p}{)}\PYG{p}{,}
            \PYG{n}{elm}\PYG{o}{.}\PYG{n}{IcPin}\PYG{p}{(}\PYG{l+s+s1}{\PYGZsq{}}\PYG{l+s+s1}{Q}\PYG{l+s+s1}{\PYGZsq{}}\PYG{p}{,} \PYG{n}{side}\PYG{o}{=}\PYG{l+s+s1}{\PYGZsq{}}\PYG{l+s+s1}{right}\PYG{l+s+s1}{\PYGZsq{}}\PYG{p}{)}\PYG{p}{,}
            \PYG{n}{elm}\PYG{o}{.}\PYG{n}{IcPin}\PYG{p}{(}\PYG{n}{side}\PYG{o}{=}\PYG{l+s+s1}{\PYGZsq{}}\PYG{l+s+s1}{B}\PYG{l+s+s1}{\PYGZsq{}}\PYG{p}{,} \PYG{n}{anchorname}\PYG{o}{=}\PYG{l+s+s1}{\PYGZsq{}}\PYG{l+s+s1}{res}\PYG{l+s+s1}{\PYGZsq{}}\PYG{p}{)}\PYG{p}{,}
        \PYG{p}{]}\PYG{p}{)}
    \PYG{n}{d} \PYG{o}{+}\PYG{o}{=} \PYG{p}{(}\PYG{n}{ff1} \PYG{o}{:=} \PYG{n}{t\PYGZus{}ff}\PYG{p}{(}\PYG{p}{)}\PYG{o}{.}\PYG{n}{anchor}\PYG{p}{(}\PYG{l+s+s1}{\PYGZsq{}}\PYG{l+s+s1}{\PYGZgt{}}\PYG{l+s+s1}{\PYGZsq{}}\PYG{p}{)}\PYG{o}{.}\PYG{n}{drop}\PYG{p}{(}\PYG{l+s+s1}{\PYGZsq{}}\PYG{l+s+s1}{nQ}\PYG{l+s+s1}{\PYGZsq{}}\PYG{p}{)}\PYG{p}{)}
    \PYG{n}{d} \PYG{o}{+}\PYG{o}{=} \PYG{n}{elm}\PYG{o}{.}\PYG{n}{Line}\PYG{p}{(}\PYG{p}{)}\PYG{o}{.}\PYG{n}{right}\PYG{p}{(}\PYG{p}{)}\PYG{o}{.}\PYG{n}{length}\PYG{p}{(}\PYG{l+m+mi}{2}\PYG{p}{)}
    \PYG{n}{d} \PYG{o}{+}\PYG{o}{=} \PYG{p}{(}\PYG{n}{ff2} \PYG{o}{:=} \PYG{n}{t\PYGZus{}ff}\PYG{p}{(}\PYG{p}{)}\PYG{o}{.}\PYG{n}{anchor}\PYG{p}{(}\PYG{l+s+s1}{\PYGZsq{}}\PYG{l+s+s1}{\PYGZgt{}}\PYG{l+s+s1}{\PYGZsq{}}\PYG{p}{)}\PYG{o}{.}\PYG{n}{drop}\PYG{p}{(}\PYG{l+s+s1}{\PYGZsq{}}\PYG{l+s+s1}{nQ}\PYG{l+s+s1}{\PYGZsq{}}\PYG{p}{)}\PYG{p}{)}

    \PYG{n}{d} \PYG{o}{+}\PYG{o}{=} \PYG{n}{elm}\PYG{o}{.}\PYG{n}{Line}\PYG{p}{(}\PYG{p}{)}\PYG{o}{.}\PYG{n}{down}\PYG{p}{(}\PYG{p}{)}\PYG{o}{.}\PYG{n}{length}\PYG{p}{(}\PYG{l+m+mi}{2}\PYG{p}{)}
    \PYG{n}{d} \PYG{o}{+}\PYG{o}{=} \PYG{n}{elm}\PYG{o}{.}\PYG{n}{Wire}\PYG{p}{(}\PYG{l+s+s1}{\PYGZsq{}}\PYG{l+s+s1}{\PYGZhy{}|}\PYG{l+s+s1}{\PYGZsq{}}\PYG{p}{)}\PYG{o}{.}\PYG{n}{to}\PYG{p}{(}\PYG{n}{ff1}\PYG{o}{.}\PYG{n}{T}\PYG{p}{)}

    \PYG{n}{d} \PYG{o}{+}\PYG{o}{=} \PYG{n}{elm}\PYG{o}{.}\PYG{n}{Line}\PYG{p}{(}\PYG{p}{)}\PYG{o}{.}\PYG{n}{left}\PYG{p}{(}\PYG{p}{)}\PYG{o}{.}\PYG{n}{at}\PYG{p}{(}\PYG{n}{ff2}\PYG{o}{.}\PYG{n}{T}\PYG{p}{)}\PYG{o}{.}\PYG{n}{length}\PYG{p}{(}\PYG{l+m+mf}{.5}\PYG{p}{)}\PYG{o}{.}\PYG{n}{dot}\PYG{p}{(}\PYG{n+nb}{open}\PYG{o}{=}\PYG{k+kc}{True}\PYG{p}{)}\PYG{o}{.}\PYG{n}{label}\PYG{p}{(}\PYG{l+s+s1}{\PYGZsq{}}\PYG{l+s+s1}{1}\PYG{l+s+s1}{\PYGZsq{}}\PYG{p}{)}

    \PYG{n}{d} \PYG{o}{+}\PYG{o}{=} \PYG{n}{elm}\PYG{o}{.}\PYG{n}{Wire}\PYG{p}{(}\PYG{l+s+s1}{\PYGZsq{}}\PYG{l+s+s1}{|\PYGZhy{}}\PYG{l+s+s1}{\PYGZsq{}}\PYG{p}{)}\PYG{o}{.}\PYG{n}{at}\PYG{p}{(}\PYG{n}{ff1}\PYG{o}{.}\PYG{n}{Q}\PYG{p}{)}\PYG{o}{.}\PYG{n}{to}\PYG{p}{(}\PYG{n}{ACmp}\PYG{o}{.}\PYG{n}{in2}\PYG{p}{)}

    \PYG{n}{d}\PYG{o}{.}\PYG{n}{here} \PYG{o}{=} \PYG{p}{(}\PYG{l+m+mi}{0}\PYG{p}{,} \PYG{o}{\PYGZhy{}}\PYG{l+m+mi}{10}\PYG{p}{)}
    \PYG{n}{d} \PYG{o}{+}\PYG{o}{=} \PYG{n}{elm}\PYG{o}{.}\PYG{n}{Line}\PYG{p}{(}\PYG{p}{)}\PYG{o}{.}\PYG{n}{idot}\PYG{p}{(}\PYG{n+nb}{open}\PYG{o}{=}\PYG{k+kc}{True}\PYG{p}{)}\PYG{o}{.}\PYG{n}{label}\PYG{p}{(}\PYG{l+s+s1}{\PYGZsq{}}\PYG{l+s+s1}{Reset}\PYG{l+s+s1}{\PYGZsq{}}\PYG{p}{)}\PYG{o}{.}\PYG{n}{length}\PYG{p}{(}\PYG{l+m+mi}{2}\PYG{p}{)}
    \PYG{n}{d}\PYG{o}{.}\PYG{n}{push}\PYG{p}{(}\PYG{p}{)}

    \PYG{n}{d} \PYG{o}{+}\PYG{o}{=} \PYG{n}{elm}\PYG{o}{.}\PYG{n}{Wire}\PYG{p}{(}\PYG{l+s+s1}{\PYGZsq{}}\PYG{l+s+s1}{\PYGZhy{}|}\PYG{l+s+s1}{\PYGZsq{}}\PYG{p}{)}\PYG{o}{.}\PYG{n}{to}\PYG{p}{(}\PYG{n}{ff1}\PYG{o}{.}\PYG{n}{res}\PYG{p}{)}
    \PYG{n}{d}\PYG{o}{.}\PYG{n}{pop}\PYG{p}{(}\PYG{p}{)}\PYG{p}{;} \PYG{n}{d}\PYG{o}{.}\PYG{n}{push}\PYG{p}{(}\PYG{p}{)}
    \PYG{n}{d} \PYG{o}{+}\PYG{o}{=} \PYG{n}{elm}\PYG{o}{.}\PYG{n}{Wire}\PYG{p}{(}\PYG{l+s+s1}{\PYGZsq{}}\PYG{l+s+s1}{\PYGZhy{}|}\PYG{l+s+s1}{\PYGZsq{}}\PYG{p}{)}\PYG{o}{.}\PYG{n}{to}\PYG{p}{(}\PYG{n}{ff2}\PYG{o}{.}\PYG{n}{res}\PYG{p}{)}
    \PYG{n}{d}\PYG{o}{.}\PYG{n}{pop}\PYG{p}{(}\PYG{p}{)}\PYG{p}{;} \PYG{n}{d}\PYG{o}{.}\PYG{n}{push}\PYG{p}{(}\PYG{p}{)}
    \PYG{n}{d} \PYG{o}{+}\PYG{o}{=} \PYG{n}{elm}\PYG{o}{.}\PYG{n}{Wire}\PYG{p}{(}\PYG{l+s+s1}{\PYGZsq{}}\PYG{l+s+s1}{\PYGZhy{}|}\PYG{l+s+s1}{\PYGZsq{}}\PYG{p}{)}\PYG{o}{.}\PYG{n}{to}\PYG{p}{(}\PYG{n}{counter}\PYG{o}{.}\PYG{n}{out1}\PYG{p}{)}

    \PYG{n}{d} \PYG{o}{+}\PYG{o}{=} \PYG{n}{elm}\PYG{o}{.}\PYG{n}{Line}\PYG{p}{(}\PYG{n}{arrow}\PYG{o}{=}\PYG{l+s+s1}{\PYGZsq{}}\PYG{l+s+s1}{|\PYGZhy{}\PYGZgt{}}\PYG{l+s+s1}{\PYGZsq{}}\PYG{p}{)}\PYG{o}{.}\PYG{n}{at}\PYG{p}{(}\PYG{n}{counter}\PYG{o}{.}\PYG{n}{out2}\PYG{p}{)}\PYG{o}{.}\PYG{n}{down}\PYG{p}{(}\PYG{p}{)}\PYG{o}{.}\PYG{n}{label}\PYG{p}{(}\PYG{l+s+s1}{\PYGZsq{}}\PYG{l+s+s1}{n}\PYG{l+s+s1}{\PYGZsq{}}\PYG{p}{,} \PYG{l+s+s1}{\PYGZsq{}}\PYG{l+s+s1}{bottom}\PYG{l+s+s1}{\PYGZsq{}}\PYG{p}{)}
\end{sphinxVerbatim}

\end{sphinxuseclass}\end{sphinxVerbatimInput}
\begin{sphinxVerbatimOutput}

\begin{sphinxuseclass}{cell_output}
\begin{sphinxVerbatim}[commandchars=\\\{\}]
D:\PYGZbs{}\PYGZus{}\PYGZus{}HTL\PYGZbs{}Dipl\PYGZbs{}CompendiumMatura\PYGZbs{}venv\PYGZbs{}Lib\PYGZbs{}site\PYGZhy{}packages\PYGZbs{}schemdraw\PYGZbs{}backends\PYGZbs{}mpl.py:292: UserWarning: Glyph 9102 (\PYGZbs{}N\PYGZob{}HYSTERESIS SYMBOL\PYGZcb{}) missing from current font.
  fig.savefig(output, format=ext, bbox\PYGZus{}inches=\PYGZsq{}tight\PYGZsq{})
\end{sphinxVerbatim}

\noindent\sphinxincludegraphics{{8c2a472123938aa07592e63b3c2a0d35c963a688c72df90158d9799ff9da6ce1}.png}

\end{sphinxuseclass}\end{sphinxVerbatimOutput}

\end{sphinxuseclass}
\begin{DUlineblock}{0em}
\item[] \sphinxstylestrong{\large Spannungsdiagramme}
\end{DUlineblock}

\sphinxAtStartPar
\sphinxincludegraphics{{Uf_diags}.png}

\begin{DUlineblock}{0em}
\item[] \sphinxstylestrong{\large Berechnung der relevanten Größe}
\end{DUlineblock}

\sphinxAtStartPar
\(N = f_x \cdot T_{ref}\)

\sphinxAtStartPar
\(f_x = \frac{T_{ref}}{N}\)

\begin{DUlineblock}{0em}
\item[] \sphinxstylestrong{\large Digitale Periodendauermessung}
\end{DUlineblock}

\begin{DUlineblock}{0em}
\item[] \sphinxstylestrong{\large Schaltung}
\end{DUlineblock}

\begin{sphinxuseclass}{cell}
\begin{sphinxuseclass}{tag_hide-input}\begin{sphinxVerbatimOutput}

\begin{sphinxuseclass}{cell_output}
\noindent\sphinxincludegraphics{{f417c9afd8f8a34fa5411e6014e882a481dcc87be3fe3d3f3ff0a779915e90d3}.png}

\end{sphinxuseclass}\end{sphinxVerbatimOutput}

\end{sphinxuseclass}
\end{sphinxuseclass}
\begin{DUlineblock}{0em}
\item[] \sphinxstylestrong{\large Spannungsdiagramme}
\end{DUlineblock}

\sphinxAtStartPar
\sphinxincludegraphics{{Uf_diags_inv}.png}

\begin{DUlineblock}{0em}
\item[] \sphinxstylestrong{\large Berechnung der relevanten Größe}
\end{DUlineblock}

\sphinxAtStartPar
\(N = T_x \cdot f_{ref}\)

\sphinxAtStartPar
\(T_x = \frac{f_{ref}}{N}\)

\begin{DUlineblock}{0em}
\item[] \sphinxstylestrong{\large Digitale Phasenverschiebung\sphinxhyphen{}Messung}
\end{DUlineblock}

\sphinxAtStartPar
\sphinxstylestrong{Anforderungen}:
\begin{itemize}
\item {} 
\sphinxAtStartPar
\(f_1 = f_2 = const\)

\item {} 
\sphinxAtStartPar
Gleichanteil = 0
HP am Eingang

\item {} 
\sphinxAtStartPar
gleiche Signalform

\end{itemize}

\begin{DUlineblock}{0em}
\item[] \sphinxstylestrong{\large Schaltung}
\end{DUlineblock}

\sphinxAtStartPar
\sphinxincludegraphics{{dphi_cir}.png}

\begin{DUlineblock}{0em}
\item[] \sphinxstylestrong{\large Spannungsdiagramme}
\end{DUlineblock}

\sphinxAtStartPar
\sphinxincludegraphics{{dphi_diags}.png}

\begin{DUlineblock}{0em}
\item[] \sphinxstylestrong{\large Berechnung der relevanten Größe}
\end{DUlineblock}

\sphinxAtStartPar
\(N = f_{ref}\cdot \Delta t\Rightarrow \Delta t = \frac{N}{f_{ref}}\)

\sphinxAtStartPar
\(\varphi = \frac{\Delta t}{T}\cdot 360° = \frac{N}{f_{ref}\cdot T}\cdot 360°\)

\begin{DUlineblock}{0em}
\item[] \sphinxstylestrong{\large U\sphinxhyphen{}f Umsetzer}
\end{DUlineblock}

\begin{DUlineblock}{0em}
\item[] \sphinxstylestrong{\large Schaltung}
\end{DUlineblock}

\sphinxAtStartPar
\sphinxincludegraphics{{uf_cur_real}.png}

\begin{DUlineblock}{0em}
\item[] \sphinxstylestrong{\large Spannungsdiagramme}
\end{DUlineblock}

\begin{sphinxuseclass}{cell}
\begin{sphinxuseclass}{tag_hide-input}\begin{sphinxVerbatimOutput}

\begin{sphinxuseclass}{cell_output}
\noindent\sphinxincludegraphics{{381a3bafdf7b5f0d4f6b189e354e463cc1fe660bc0dd1dd914e3164b76fdd7d5}.png}

\end{sphinxuseclass}\end{sphinxVerbatimOutput}

\end{sphinxuseclass}
\end{sphinxuseclass}
\begin{DUlineblock}{0em}
\item[] \sphinxstylestrong{\large Berechnung der relevanten Größe}
\end{DUlineblock}

\sphinxAtStartPar
\(f = \frac{1}{T} = \frac{1}{2t} = \frac{U_m}{4R_4 C \alpha U_{a, max}}\)

\begin{DUlineblock}{0em}
\item[] \sphinxstylestrong{\large Anwendungsbeispiele}
\end{DUlineblock}

\begin{DUlineblock}{0em}
\item[] \sphinxstylestrong{\large DMS\sphinxhyphen{}Messbrücke}
\end{DUlineblock}

\begin{DUlineblock}{0em}
\item[] \sphinxstylestrong{\large Schaltungen}
\end{DUlineblock}

\begin{sphinxuseclass}{cell}
\begin{sphinxuseclass}{tag_hide-input}\begin{sphinxVerbatimOutput}

\begin{sphinxuseclass}{cell_output}
\noindent\sphinxincludegraphics{{8ed85907dbedc3d28b88f6e02beeb48870bc8a7e44250af57308fd89e55408a3}.png}

\end{sphinxuseclass}\end{sphinxVerbatimOutput}

\end{sphinxuseclass}
\end{sphinxuseclass}
\begin{DUlineblock}{0em}
\item[] \sphinxstylestrong{\large OPV\sphinxhyphen{}Verstärkerschaltung}
\end{DUlineblock}

\sphinxAtStartPar
\sphinxincludegraphics{{differential_amp}.png}

\begin{DUlineblock}{0em}
\item[] \sphinxstylestrong{\large Berechnung/Herleitung der Ausgangsgröße}
\end{DUlineblock}

\sphinxAtStartPar
\(U_{1} = U_B = U_{e2} - U_{e1} \wedge\)

\sphinxAtStartPar
\(I_{12} = \frac{U_1}{R_1} = \frac{U_{Diff}}{R_1 + 2R_2}\wedge\)

\sphinxAtStartPar
\(U_a = (U'_{e1} - U'_{e2}) \Rightarrow U_a = U_diff\)

\sphinxAtStartPar
\(\therefore U_a = \left(1 + \frac{2R_2}{R_1}\right)\cdot (U_{e2} - U_{e2})\)…Differenz der Potentiale zwischen 2 Punkten

\begin{DUlineblock}{0em}
\item[] \sphinxstylestrong{\large Biasstromkompensation}
\end{DUlineblock}

\sphinxAtStartPar
OPVs sind nicht ideal ( Eingangsströme ) beeinflussen die Verstärkung deshalb Kompensation durch extra Schaltung.

\begin{DUlineblock}{0em}
\item[] \sphinxstylestrong{\large Signalaufbereitung}
\end{DUlineblock}

\begin{DUlineblock}{0em}
\item[] \sphinxstylestrong{\large Digitale Verarbeitungskette}
\end{DUlineblock}

\begin{DUlineblock}{0em}
\item[] \sphinxstylestrong{\large Anti Aliasing Filter}
\end{DUlineblock}

\sphinxAtStartPar
Um zu verhindern, dass das Abtasttheorem verletzt wird, werden Anti\sphinxhyphen{}Aliasing Filter verwendet.
Diese verhindern,
dass die Signalfrequenz höher ist als die Maximal erlaubte

\begin{sphinxuseclass}{cell}
\begin{sphinxuseclass}{tag_hide-input}\begin{sphinxVerbatimOutput}

\begin{sphinxuseclass}{cell_output}
\noindent\sphinxincludegraphics{{8792b515c67a9906770d665e216b26d2c81c410036e7be1be3282f74207d060b}.png}

\end{sphinxuseclass}\end{sphinxVerbatimOutput}

\end{sphinxuseclass}
\end{sphinxuseclass}
\begin{DUlineblock}{0em}
\item[] \sphinxstylestrong{S\&H Glied}
\end{DUlineblock}

\sphinxAtStartPar
Für die AD\sphinxhyphen{}Umwandlung muss das Eingangssignal konstant Gehalten werden.
Dafür werden \sphinxstylestrong{S}ample und \sphinxstylestrong{H}old Glieder verwendet

\begin{sphinxuseclass}{cell}
\begin{sphinxuseclass}{tag_hide-input}\begin{sphinxVerbatimOutput}

\begin{sphinxuseclass}{cell_output}
\noindent\sphinxincludegraphics{{d5654f64a8e05c6d74b9008c60c15d3095adaef82ded5c9751ba9172a0f45fed}.png}

\end{sphinxuseclass}\end{sphinxVerbatimOutput}

\end{sphinxuseclass}
\end{sphinxuseclass}
\begin{DUlineblock}{0em}
\item[] \sphinxstylestrong{\large Abtasttheorem}
\end{DUlineblock}

\sphinxAtStartPar
Die Abtastfrequenz muss mindestens doppelt so hoch sein wie die Signalfrequenz.

\sphinxAtStartPar
Wenn das Abtasttheorem verletzt wird,
so werden die hohen Frequenzanteile als niedrigere aufgefasst,
welche das Signal verzerren

\begin{DUlineblock}{0em}
\item[] \sphinxstylestrong{\large Umsetzungskennlinien}
\end{DUlineblock}

\begin{DUlineblock}{0em}
\item[] \sphinxstylestrong{\large AD\sphinxhyphen{}Wandler}
\end{DUlineblock}

\begin{DUlineblock}{0em}
\item[] \sphinxstylestrong{\large Sukzessive Approximationsverfahren}
\end{DUlineblock}

\begin{DUlineblock}{0em}
\item[] \sphinxstylestrong{\large Schaltung}
\end{DUlineblock}

\begin{sphinxuseclass}{cell}
\begin{sphinxuseclass}{tag_hide-input}
\end{sphinxuseclass}
\end{sphinxuseclass}
\begin{sphinxuseclass}{cell}
\begin{sphinxuseclass}{tag_hide-input}
\end{sphinxuseclass}
\end{sphinxuseclass}
\begin{sphinxuseclass}{cell}
\begin{sphinxuseclass}{tag_hide-input}
\end{sphinxuseclass}
\end{sphinxuseclass}
\begin{sphinxuseclass}{cell}
\begin{sphinxuseclass}{tag_hide-input}\begin{sphinxVerbatimOutput}

\begin{sphinxuseclass}{cell_output}
\noindent\sphinxincludegraphics{{5e436d2afff3d4898702a5de0aca4baccd0280ec3d3a1747b03ed55241da7483}.png}

\end{sphinxuseclass}\end{sphinxVerbatimOutput}

\end{sphinxuseclass}
\end{sphinxuseclass}
\begin{DUlineblock}{0em}
\item[] \sphinxstylestrong{\large Funktionsprinzip}
\end{DUlineblock}
\begin{enumerate}
\sphinxsetlistlabels{\arabic}{enumi}{enumii}{}{.}%
\item {} 
\sphinxAtStartPar
Die Bits des SAR (\sphinxstylestrong{S}ukkzessive *\sphinxstylestrong{A}pprox \sphinxstylestrong{R}egister) sind alle auf 0
Man hat einen Zeiger auf ein Bit,
welches am Anfang auf das \sphinxstylestrong{MSB} zeigt.

\item {} 
\sphinxAtStartPar
Das Bit des Zeigers wird auf 1 gesetzt,

\item {} 
\sphinxAtStartPar
Über einen DA\sphinxhyphen{}Wandler wird der Ausgang wieder zu einer analogen Spannung gewandelt.

\item {} 
\sphinxAtStartPar
Wenn die Spannung nun größer als die Eingangsspannung ist,
wird das Bit wieder auf 0 gesetzt
Ansonsten bleibt es auf 1.

\item {} 
\sphinxAtStartPar
Wenn der Zeiger noch nicht das \sphinxstylestrong{LSB} erreicht hat,
geht er um eine stelle zum nächsten weniger Werten Bit.
Wenn der Zeiger das LSB erreicht hat,
so ist die Wandlung beendet und das \sphinxstyleemphasis{Finished}\sphinxhyphen{}Flag wird auf 1 gesetzt

\end{enumerate}

\sphinxAtStartPar
Laufzeit: \(O(log~n)\)

\begin{DUlineblock}{0em}
\item[] \sphinxstylestrong{\large Diagramme}
\end{DUlineblock}

\sphinxAtStartPar
Beispiel mit n=4 Bit und \(U_e = 10.7V\), der Wert des LSB Beträgt \(1V\) (MSB=> \(8V\))

\begin{sphinxuseclass}{cell}
\begin{sphinxuseclass}{tag_hide-input}\begin{sphinxVerbatimOutput}

\begin{sphinxuseclass}{cell_output}
\noindent\sphinxincludegraphics{{c73b3bcfb6ccc8e2bc1c77dbfae7becbf7eaad2039df350b9815f8aafe2a0c17}.png}

\end{sphinxuseclass}\end{sphinxVerbatimOutput}

\end{sphinxuseclass}
\end{sphinxuseclass}
\begin{DUlineblock}{0em}
\item[] \sphinxstylestrong{\large Herleitungen}
\end{DUlineblock}

\sphinxAtStartPar
\(U(z) = U_{ref}\cdot\frac{t}{t_{max} +1}\cdot U_e\)\\
\(Z = \frac{Z_{max} + 1}{U_{ref}} \cdot U_e\)

\begin{DUlineblock}{0em}
\item[] \sphinxstylestrong{\large Single Slope:}
\end{DUlineblock}

\begin{DUlineblock}{0em}
\item[] \sphinxstylestrong{\large Schaltung}
\end{DUlineblock}

\sphinxAtStartPar
\sphinxincludegraphics{{single_slope_cir}.png}

\sphinxAtStartPar
\sphinxstyleemphasis{Bei uns S\&H Glied am Eingang}
\sphinxstyleemphasis{= kann bei uns ein \& sein}

\begin{DUlineblock}{0em}
\item[] \sphinxstylestrong{\large Funktionsprinzip}
\end{DUlineblock}
\begin{itemize}
\item {} 
\sphinxAtStartPar
Es wird ein Sägezahn mit dem Eingang und mit Ground verglichen.

\item {} 
\sphinxAtStartPar
Wenn der Sägezahn größer 0 ist und kleiner als das Eingangssignals
so ist das Und Gatter der Beiden Komparatoren HIGH.
Durch ein UND Gatter mit einem Clock Signal,
wird dieses nur in diesen Zeitraum durchgelassen.

\item {} 
\sphinxAtStartPar
Bei jeder durchgelassenen Clock\sphinxhyphen{}Flanke zählt ein Counter nach oben.

\end{itemize}

\begin{DUlineblock}{0em}
\item[] \sphinxstylestrong{\large Diagramme}
\end{DUlineblock}

\begin{sphinxuseclass}{cell}
\begin{sphinxuseclass}{tag_hide-input}\begin{sphinxVerbatimOutput}

\begin{sphinxuseclass}{cell_output}
\begin{sphinxVerbatim}[commandchars=\\\{\}]
Text(0, 0.5, \PYGZsq{}\PYGZdl{}U\PYGZus{}y\PYGZdl{}\PYGZsq{})
\end{sphinxVerbatim}

\noindent\sphinxincludegraphics{{427591d3452bfa8fd437fde8509e719280476d9c552fb1b75ad1c53470182993}.png}

\end{sphinxuseclass}\end{sphinxVerbatimOutput}

\end{sphinxuseclass}
\end{sphinxuseclass}
\begin{DUlineblock}{0em}
\item[] \sphinxstylestrong{\large Herleitungen}
\end{DUlineblock}

\sphinxAtStartPar
\(N = \Delta t \cdot f_{clk}\)

\begin{DUlineblock}{0em}
\item[] \sphinxstylestrong{\large Dual Slope:}
\end{DUlineblock}

\begin{DUlineblock}{0em}
\item[] \sphinxstylestrong{\large Schaltung}
\end{DUlineblock}

\sphinxAtStartPar
\sphinxincludegraphics{{dual_slope_cir}.png}

\begin{DUlineblock}{0em}
\item[] \sphinxstylestrong{\large Funktionsprinzip}
\end{DUlineblock}
\begin{itemize}
\item {} 
\sphinxAtStartPar
Es wird nach der Eingangsspannung invers für die Zeit \(t_1\) Integriert

\item {} 
\sphinxAtStartPar
Danach wird nach einer Referenzspannung nach oben integriert (\(-\)\&\(-\) \(\Rightarrow~~+\))

\item {} 
\sphinxAtStartPar
Während des 2. Integrierens zählt ein Counter nach oben,
dieser Vorgang wird abgebrochen, wenn der Integrierte Wert \(~0\) erreicht (Komparator)

\end{itemize}

\sphinxAtStartPar
\sphinxstylestrong{WICHTIG:} \(t_1\) ist konstant

\sphinxAtStartPar
Da die Flächen gleich sind, fallen die \(RC\) Komponenten weg, wodurch man nicht von Bauteildriften betroffen ist.

\begin{DUlineblock}{0em}
\item[] \sphinxstylestrong{\large Diagramme}
\end{DUlineblock}

\sphinxAtStartPar
\sphinxincludegraphics{{dual_slope_diag}.png}

\begin{DUlineblock}{0em}
\item[] \sphinxstylestrong{\large Herleitungen}
\end{DUlineblock}

\sphinxAtStartPar
Die Fläche unter beiden Analogwerten ist gleich.

\sphinxAtStartPar
\(-\frac{1}{RC} \int_0^{t_1} U_e dt = -\frac{1}{RC} \int_{t_1}^{t_2 + t_1} U_{ref}dt\)\\
\(U_e\cdot t_1 = U_{ref}\cdot t_1 + U_{ref}\cdot t_2 - U_{ref}\cdot t_1\)\\
\(U_e\cdot t_1 = U_{ref} \cdot t_2\)\\
\(t_1 = 2^n\cdot T_{clk} \wedge t_2 \cdot Z\cdot T_{clk}\)\\
\(U_e \cdot 2^n \cdot T_{clk} = U_{ref}\cdot Z \cdot T_{clk}\)\\
\(Z = \frac{U_e}{U_{ref}} \cdot (Z_{max} + 1)\)

\begin{DUlineblock}{0em}
\item[] \sphinxstylestrong{\large Zählverfahren}
\end{DUlineblock}

\begin{DUlineblock}{0em}
\item[] \sphinxstylestrong{\large Schaltung}
\end{DUlineblock}

\begin{sphinxuseclass}{cell}
\begin{sphinxuseclass}{tag_hide-input}
\end{sphinxuseclass}
\end{sphinxuseclass}
\begin{sphinxuseclass}{cell}
\begin{sphinxuseclass}{tag_hide-input}
\end{sphinxuseclass}
\end{sphinxuseclass}
\begin{sphinxuseclass}{cell}
\begin{sphinxuseclass}{tag_hide-input}
\end{sphinxuseclass}
\end{sphinxuseclass}
\begin{sphinxuseclass}{cell}
\begin{sphinxuseclass}{tag_hide-input}\begin{sphinxVerbatimOutput}

\begin{sphinxuseclass}{cell_output}
\noindent\sphinxincludegraphics{{0e0b797393e3c82f06747706a5a206add69bdff368b625cb71cc250575e052cc}.png}

\end{sphinxuseclass}\end{sphinxVerbatimOutput}

\end{sphinxuseclass}
\end{sphinxuseclass}
\begin{DUlineblock}{0em}
\item[] \sphinxstylestrong{\large Funktionsprinzip}
\end{DUlineblock}

\sphinxAtStartPar
Solange der DA gewandelte wert von Z kleiner als U\_e ist (Komparator gibt \(0\) aus),
zählt der Counter nach oben.
Wenn der Wert größer als U\_e ist (Komparator gibt \(0\) aus).
so zählt der Counter nach unten.
Laufzeit: \(O(2^n)\)

\sphinxAtStartPar
Wenn sich der Eingang um weniger als \(U_{LSB}\cdot f_{clk} \equiv \frac{[V]}{[s]}\) ändert,
so kann das S\&H Glied weggelassen werden.
Dadurch folgt der Ausgangswert dem Eingangswert.

\begin{DUlineblock}{0em}
\item[] \sphinxstylestrong{\large Diagramme}
\end{DUlineblock}

\begin{sphinxuseclass}{cell}
\begin{sphinxuseclass}{tag_hide-input}\begin{sphinxVerbatimOutput}

\begin{sphinxuseclass}{cell_output}
\begin{sphinxVerbatim}[commandchars=\\\{\}]
[\PYGZlt{}matplotlib.lines.Line2D at 0x2b4b973b610\PYGZgt{}]
\end{sphinxVerbatim}

\noindent\sphinxincludegraphics{{498309499adeb8b1b717c5587cb3db95973c5bb3cfcabaa03367027efba823f3}.png}

\end{sphinxuseclass}\end{sphinxVerbatimOutput}

\end{sphinxuseclass}
\end{sphinxuseclass}
\begin{DUlineblock}{0em}
\item[] \sphinxstylestrong{\large DA\sphinxhyphen{}Wandler}
\end{DUlineblock}

\begin{DUlineblock}{0em}
\item[] \sphinxstylestrong{\large R2R\sphinxhyphen{}Netzwerk}
\end{DUlineblock}

\begin{sphinxuseclass}{cell}
\begin{sphinxuseclass}{tag_hide-input}\begin{sphinxVerbatimOutput}

\begin{sphinxuseclass}{cell_output}
\noindent\sphinxincludegraphics{{eacfbbb2c1001385890e20dd56debaef1de171ffd1c56f7f8a2823d23475ad29}.png}

\end{sphinxuseclass}\end{sphinxVerbatimOutput}

\end{sphinxuseclass}
\end{sphinxuseclass}
\begin{DUlineblock}{0em}
\item[] \sphinxstylestrong{\large Prinzip der gewichteten Ströme}
\end{DUlineblock}

\sphinxAtStartPar
Nach Überlagerungs\sphinxhyphen{}Prinzip Summe von jeden einzelnen Pfad.

\sphinxAtStartPar
\(U_a = -U_{ref}\cdot \left(Z_3\cdot\frac{R_F}{2R} + Z_2\cdot\frac{R_F}{4R} + Z_1\cdot\frac{R_F}{8R} + Z_0\cdot\frac{R_F}{16R}\right)\)\\
\(U_a = -U_{ref} \cdot \frac{R_F}{16R}\cdot (8\cdot Z_3 + 4\cdot Z_2 + 2 \cdot Z_1 + Z_0)\)\\
In dieser Form gut einsehbar, jeder Schalter repräsentiert ein Bit.\\
\(\forall Z_i \in \{0, 1\}\)\\
\(U_a = -U_{ref} \cdot\frac{R_F}{16R}\cdot Z = -U_{ref}\cdot\frac{R_F}{R}\cdot \frac{Z}{Z_{max}  +1}\)

\sphinxAtStartPar
Strom ist unabhängig von Z.\\
\(I'  = U_{ref} \cdot \frac{Z}{Z_{max} + 1}\cdot \frac{1}{R}\)\\
\(I'' = \frac{U_{ref}}{R}\cdot \frac{Z_{max} - Z}{Z_{max} + 1}\)

\sphinxAtStartPar
\(I = I' + I''\)\\
nicht von \(Z\) abhängig

\begin{DUlineblock}{0em}
\item[] \sphinxstylestrong{\large Inverses R2R\sphinxhyphen{}Netzwerk}
\end{DUlineblock}

\begin{sphinxuseclass}{cell}
\begin{sphinxuseclass}{tag_hide-input}\begin{sphinxVerbatimOutput}

\begin{sphinxuseclass}{cell_output}
\noindent\sphinxincludegraphics{{88cd0f2d34d9c951292fe4e015eba010d1bebe095167548e22f1c8263968cbea}.png}

\end{sphinxuseclass}\end{sphinxVerbatimOutput}

\end{sphinxuseclass}
\end{sphinxuseclass}
\sphinxAtStartPar
Bei Überlagerung alle Werte bis zum Schalter kollabieren zu 2R

\sphinxAtStartPar
Durch Teilung danach,
\(\varphi_+ = Z_3 \cdot \frac{U_{ref}}{3} + Z_2 \cdot \frac{U_{ref}}{6} + Z_1 \cdot \frac{U_{ref}}{12} + Z_0 \cdot \frac{U_{ref}}{24}\)\\
\(\varphi_+ = \frac{U_{ref}}{24} \cdot (8 \cdot Z_3 + 4 \cdot Z_2 +  2 \cdot Z_1 + Z_0)\)\\
\(\varphi_+ = \frac{U_{ref}}{24}\cdot Z = \frac{16}{16}\cdot \frac{U_{ref}}{24}\cdot Z\)

\sphinxAtStartPar
\(U_a = \frac{U_{ref}}{3}\cdot \frac{Z}{Z_{max}} \cdot \left(1 + \frac{R_2}{R_1} \right)\)

\begin{DUlineblock}{0em}
\item[] \sphinxstylestrong{\large Industrielle Anwendung}
\end{DUlineblock}

\begin{DUlineblock}{0em}
\item[] \sphinxstylestrong{\large Bustopologie und Zugriffsverfahren}
\end{DUlineblock}

\begin{DUlineblock}{0em}
\item[] \sphinxstylestrong{\large Bustopologievarianten}
\end{DUlineblock}
\begin{itemize}
\item {} 
\sphinxAtStartPar
Stern

\item {} 
\sphinxAtStartPar
Ring

\item {} 
\sphinxAtStartPar
Peer\sphinxhyphen{}To\sphinxhyphen{}Peer

\item {} 
\sphinxAtStartPar
Bus

\item {} 
\sphinxAtStartPar
Baum

\end{itemize}

\begin{DUlineblock}{0em}
\item[] \sphinxstylestrong{\large Master\sphinxhyphen{}Slave\sphinxhyphen{}Prinzip}
\end{DUlineblock}

\sphinxAtStartPar
In einem Netzwerk gibt es einen oder mehrere Teilnehmer (Master), welche die anderen Teilnehmer ansprechen
und den Datenaustausch steuern.

\begin{DUlineblock}{0em}
\item[] \sphinxstylestrong{\large CSMA/CD}
\end{DUlineblock}

\sphinxAtStartPar
Hier wird ein Störsignal ausgesendet sobald eine Kollision entdeckt wurde.
Alle Teilnehmer stoppen zu senden und berechnen sich Zufallszahlen,
welche bestimmen wie lange sie warten bis sie wieder versuchen die Nachricht zu versenden.

\sphinxAtStartPar
Deshalb ist diese Art \sphinxstylestrong{nicht} Deterministisch!

\begin{DUlineblock}{0em}
\item[] \sphinxstylestrong{\large CSMA/CA}
\end{DUlineblock}

\sphinxAtStartPar
Hier bekommt jeder Teilnehmer eine Adresse,
je nach der Wertigkeit diese Adresse wird bestimmt wer senden darf.
Denn sobald ein Teilnehmer auf diese Leitung überschrieben wird, so schaltet sich dieser weg.

\begin{sphinxuseclass}{cell}
\begin{sphinxuseclass}{tag_hide-input}\begin{sphinxVerbatimOutput}

\begin{sphinxuseclass}{cell_output}
\begin{sphinxVerbatim}[commandchars=\\\{\}]
\PYGZsq{}\PYGZsq{}
\end{sphinxVerbatim}

\noindent\sphinxincludegraphics{{30071a1cd1b56f6de1cb809a1df57515f29cc5c33331a7f9f89df7aa8948af7d}.png}

\end{sphinxuseclass}\end{sphinxVerbatimOutput}

\end{sphinxuseclass}
\end{sphinxuseclass}
\begin{DUlineblock}{0em}
\item[] \sphinxstylestrong{\large Wired\sphinxhyphen{}AND}
\end{DUlineblock}

\sphinxAtStartPar
Die Ausgänge sind Open\sphinxhyphen{}Collectors und die Leitung wird auf HIGH gezogen.
Sobald nun mindestens ein Teilnehmer die Leitung auf LOW ziehen möchte geht die Leitung auf LOW.

\sphinxAtStartPar
Somit müssen alle Teilnehmer HIGH schreiben wollen damit auch HIGH auf die Leitung geschrieben wird,
Diese Eigenschaft führt dazu, dass wenn ein Teilnehmer überschrieben wird,
sich dieser wegschaltet und somit CSMA\sphinxhyphen{}CA ermöglicht.

\begin{DUlineblock}{0em}
\item[] \sphinxstylestrong{\large Serielle Schnittstelle}
\end{DUlineblock}

\begin{DUlineblock}{0em}
\item[] \sphinxstylestrong{\large RS232}
\end{DUlineblock}

\begin{DUlineblock}{0em}
\item[] \sphinxstylestrong{\large Eigenschaften}
\end{DUlineblock}
\begin{itemize}
\item {} 
\sphinxAtStartPar
Ruhepegel: HIGH


\begin{savenotes}\sphinxattablestart
\centering
\begin{tabulary}{\linewidth}[t]{|T|T|T|}
\hline

\sphinxAtStartPar

&\sphinxstyletheadfamily 
\sphinxAtStartPar
RS232
&\sphinxstyletheadfamily 
\sphinxAtStartPar
UART
\\
\hline
\sphinxAtStartPar
LOW
&
\sphinxAtStartPar
\(~+3~...~+15V\)
&
\sphinxAtStartPar
\(0V\)
\\
\hline
\sphinxAtStartPar
HIGH
&
\sphinxAtStartPar
\(-3~...~-15V\)
&
\sphinxAtStartPar
\(5V\)
\\
\hline
\end{tabulary}
\par
\sphinxattableend\end{savenotes}

\item {} 
\sphinxAtStartPar
5\sphinxhyphen{}8 Datenbits\\
LSB als erstes\\
MSB als letztes

\item {} 
\sphinxAtStartPar
0\sphinxhyphen{}1 Paritätsbit\\
EVEN oder ODD

\item {} 
\sphinxAtStartPar
1, 1.5, 2 Stop Bits
Ruhephase bis zur nächsten Datenübertragung

\end{itemize}

\sphinxAtStartPar
Baudrate = \(Bit/s\)\\
bei RS232 vielfaches von \(150\)

\begin{DUlineblock}{0em}
\item[] \sphinxstylestrong{\large Störeinflüsse}
\end{DUlineblock}

\begin{DUlineblock}{0em}
\item[] \sphinxstylestrong{\large Gleichtaktstörung}
\end{DUlineblock}

\sphinxAtStartPar
Bei kapazitiver Störung wird das Potential verändert
Bei differentialer Datenübertragung werden beide Leitungen beinahe gleich gestört
und die Potentialdifferenz bleibt nahezu gleich

\begin{DUlineblock}{0em}
\item[] \sphinxstylestrong{\large Induktive Störung}
\end{DUlineblock}

\sphinxAtStartPar
Durch Leitungsschleifen treten induktive Störungen auf.
Wenn die Leitung verdrillt wird (twisted pair), so gibt es mehrere kleinere Felder,
welche ein gegengleiches Vorzeichen und somit die induktive Störung minimal halten.

\begin{DUlineblock}{0em}
\item[] \sphinxstylestrong{\large Differenzielle Datenübertragung}
\end{DUlineblock}

\sphinxAtStartPar
Es wird nicht das Potenzial der Leitung gemessen,
sondern die Potentialunterschied zwischen zwei Leitungen.\\
Am Empfänger wird das Signal dekodiert (Optokoppler)

\begin{DUlineblock}{0em}
\item[] \sphinxstylestrong{\large MAX\sphinxhyphen{}232\sphinxhyphen{}Ladungspumpenprinzip}
\end{DUlineblock}

\begin{DUlineblock}{0em}
\item[] \sphinxstylestrong{\large Ladungspumpe}
\end{DUlineblock}

\begin{sphinxuseclass}{cell}
\begin{sphinxuseclass}{tag_hide-input}\begin{sphinxVerbatimOutput}

\begin{sphinxuseclass}{cell_output}
\noindent\sphinxincludegraphics{{3a9d412a47c2f3cfc9c8cd95a7aa5d52bb4a95ed700ea55097f81466081036a5}.png}

\end{sphinxuseclass}\end{sphinxVerbatimOutput}

\end{sphinxuseclass}
\end{sphinxuseclass}
\begin{sphinxuseclass}{cell}
\begin{sphinxuseclass}{tag_hide-input}\begin{sphinxVerbatimOutput}

\begin{sphinxuseclass}{cell_output}
\noindent\sphinxincludegraphics{{e631dad93558a3b076ecd448ae0ca3e3e5f969bac1d30370ad48a256f5a78ff8}.png}

\noindent\sphinxincludegraphics{{a8efec7fb9b3920276c2b02703597a1622d0b431f9e58a82115b1e5e7134f48e}.png}

\noindent\sphinxincludegraphics{{1076a29f4a29d0113b41053eedee9e626bd6f18c91e45e27ab31b39daa429723}.png}

\end{sphinxuseclass}\end{sphinxVerbatimOutput}

\end{sphinxuseclass}
\end{sphinxuseclass}
\begin{DUlineblock}{0em}
\item[] \sphinxstylestrong{\large Spannunginvertierer}
\end{DUlineblock}

\begin{sphinxuseclass}{cell}
\begin{sphinxuseclass}{tag_hide-input}\begin{sphinxVerbatimOutput}

\begin{sphinxuseclass}{cell_output}
\noindent\sphinxincludegraphics{{52a311aadf7168922c759cc88048509ddf621134cdd977094c8b373e01e50bea}.png}

\end{sphinxuseclass}\end{sphinxVerbatimOutput}

\end{sphinxuseclass}
\end{sphinxuseclass}
\begin{sphinxuseclass}{cell}
\begin{sphinxuseclass}{tag_hide-input}\begin{sphinxVerbatimOutput}

\begin{sphinxuseclass}{cell_output}
\noindent\sphinxincludegraphics{{aaaa9ea03954649616c304437998e9fae246eb699a022d651c5e457231a339af}.png}

\noindent\sphinxincludegraphics{{546d558f8b4d6c013bec3d51b0a5a845928a6f94492f2fda4ca9a348d44316bc}.png}

\noindent\sphinxincludegraphics{{7fd4305ef0d3b337e825357c3ab43697667030c00ce743a2aa3e302b9b473f46}.png}

\end{sphinxuseclass}\end{sphinxVerbatimOutput}

\end{sphinxuseclass}
\end{sphinxuseclass}
\begin{DUlineblock}{0em}
\item[] \sphinxstylestrong{\large Datensicherungsverfahren}
\end{DUlineblock}

\begin{DUlineblock}{0em}
\item[] \sphinxstylestrong{\large Paritätsbit}
\end{DUlineblock}

\sphinxAtStartPar
Es wird die Anzahl der Einsen gezählt,
anschließend wird aufgrund der Anzahl bestimmt welchen Wert das Paritätsbit haben soll.\\
Das Paritätsbit wird XOR gebildet
XOR können nacheinander gerechnet werden.
\begin{itemize}
\item {} 
\sphinxAtStartPar
EVEN
Mit dem Paritätsbit ist die Anzahl der Einen Gerade\\
\(\quad\left(b_0 \oplus b_1 \oplus b_2 \oplus ...\oplus b_n\right)\)

\item {} 
\sphinxAtStartPar
ODD
Mit dem Paritätsbit ist die Anzahl der Einsen Ungerade\\
\(\neg\left(b_0 \oplus b_1 \oplus b_2 \oplus ...\oplus b_n\right)\)

\end{itemize}

\sphinxAtStartPar
Zur Kontrolle wird beim Empfänger wieder die Parität gebildet
und anschließend mit dem gesendeten Paritätsbit XOR gerechnet.
Wenn dabei \(0\) rauskommt ist die Nachricht OK (sofern max 1 Bit\sphinxhyphen{}Fehler auftreten können)
und wenn \(1\) rauskommt ist die Nachricht sicher Falsch

\sphinxAtStartPar
Das Paritätsbit hat eine Hamming Distanz von \(1\)

\begin{DUlineblock}{0em}
\item[] \sphinxstylestrong{\large Hamming\sphinxhyphen{}Distanz}
\end{DUlineblock}

\sphinxAtStartPar
Die Hamming Distanz gibt an wie viele Bit\sphinxhyphen{}Flips mindestens auftreten müssen,
um zum nächsten gültigen Wert zu kommen.
\begin{itemize}
\item {} 
\sphinxAtStartPar
max. detektierbare Fehler Anzahl:\\
\(k = h-1\)

\item {} 
\sphinxAtStartPar
max. korrigierbare Fehler Anzahl:\\
\(t = \text{floor}\left(\frac{d_{min} - 1}{2}\right)\)

\end{itemize}

\begin{DUlineblock}{0em}
\item[] \sphinxstylestrong{\large CRC (Cycle Redundancy Check)}
\end{DUlineblock}

\begin{DUlineblock}{0em}
\item[] \sphinxstylestrong{\large Grundprinzip}
\end{DUlineblock}

\sphinxAtStartPar
\sphinxincludegraphics{{crc_principle}.png}

\begin{DUlineblock}{0em}
\item[] \sphinxstylestrong{\large Generatorpolynom}
\end{DUlineblock}

\sphinxAtStartPar
\(I(x)\)…Nutzdaten\\
\(G(x)\)…Generatorpolynom\\
\(R(x)\)…Divisionsrestes

\sphinxAtStartPar
Generatorpolynom gibt eine Bitfolge an,
alle vorhandenen Potenzen geben eine 1 in der Bitfolge an\\
Das MSB \sphinxstylestrong{muss} immer 1 sein.\\
Das Generatorpolynom hat die Höchste Potenz + 1 stellen.

\sphinxAtStartPar
Bsp.\\
\(G(x) = x^5 + x^4 + x^2 + 1\)\\
\(G(x) = 110101\)

\sphinxAtStartPar
Berechnungsschema:\\
\(I(x)\qquad r-1~\text{Nullen}\quad:\quad G(x)\)

\begin{DUlineblock}{0em}
\item[] \sphinxstylestrong{\large Rechnenprinzip}
\end{DUlineblock}

\sphinxAtStartPar
\sphinxincludegraphics{{crc_calc}.png}

\begin{DUlineblock}{0em}
\item[] \sphinxstylestrong{\large Fehlererkennung}
\end{DUlineblock}

\sphinxAtStartPar
Zur Kontrolle muss die CRC\sphinxhyphen{}Checksum erneut berechnet werden,
mit der gesendeten Checksumme als Teil der Übertragung

\sphinxAtStartPar
Wenn die Übertragung korrekt und ohne Fehler war,
ist der Wert welcher zurück gesendet wird 0

\sphinxAtStartPar
\sphinxincludegraphics{{crc_test}.png}

\begin{DUlineblock}{0em}
\item[] \sphinxstylestrong{\large Schaltung}
\end{DUlineblock}

\sphinxAtStartPar
\sphinxincludegraphics{{crc_circ}.png}

\begin{DUlineblock}{0em}
\item[] \sphinxstylestrong{\large I2C}
\end{DUlineblock}

\begin{DUlineblock}{0em}
\item[] \sphinxstylestrong{\large Spezifikation}
\end{DUlineblock}

\sphinxAtStartPar
Es gibt zwei Leitungen SDA (Datenleitung) und SCL (Clockleitung),
welche auf HIGH gezogen werden.\\
Dadurch ist die Leitung auf 0,
sobald mindestens ein Teilnehmer die Leitung auf 0 zieht.

\sphinxAtStartPar
Jeder Teilnehmer besitzt eine einzigartige Adresse,
je nach Aufgabe kann dieser nur Senden (Tastatur),
nur Empfangen (LC\sphinxhyphen{}interface) oder beides(Speicherbaustein)\\
Der Master muss beim Senden die Arbitrierung “gewinnen”,
in der Arbitrierungsphase wird die Empfänger Adresse ausgeschrieben.
Der Teilnehmer der die niederwertigste Adresse ansprechen will gewinnt die Arbitrierung.\\
Nur weil jeder Teilnehmer Master und Slave sein könnte,
muss dies nicht sein.

\sphinxAtStartPar
Wenn sich ein Bit\sphinxhyphen{}Level ändert muss die SCL\sphinxhyphen{}Leitung auf Low sein
Ferner wenn ein Teilnehmer länger braucht um etwas zu verarbeiten kann dieser die SCL Leitung auf LOW ziehen
und somit mit der Übertragung fortfahren wenn der Teilnehmer bereit ist.

\sphinxAtStartPar
Start\sphinxhyphen{} und Stopp\sphinxhyphen{}Bedingungen werden durch Änderung des SDA Pegel bei HIGH Pegel auf der SCL Leitung gesendet (Verletzung des nicht Veränderns der SDA\sphinxhyphen{}Leitung)

\begin{DUlineblock}{0em}
\item[] \sphinxstylestrong{\large Adressierung}
\end{DUlineblock}

\sphinxAtStartPar
Jeder Teilnehmer, welcher angesprochen werden soll,
hat eine eindeutige Slave Adresse,
welche ausgesendet wird, wenn jemand diesen Baustein ansprechen will.
Master\sphinxhyphen{}Only Bausteine brauchen keine Adresse.

\sphinxAtStartPar
Die Adresse ist normalerweise 7\sphinxhyphen{}Bit, kann aber auf 10\sphinxhyphen{}Bit erweitert werden (meist 7\sphinxhyphen{}Bit).\\
Diese Adresse wird bei der Arbitrierung ausgesendet,
Zusätzlich zur Adresse wird noch ein Read Flag ausgesendet \(R/\overline{W}\),
welche indiziert ob der aus dem Baustein gelesen oder geschrieben werden soll.

\begin{sphinxuseclass}{cell}
\begin{sphinxuseclass}{tag_hide-input}\begin{sphinxVerbatimOutput}

\begin{sphinxuseclass}{cell_output}
\noindent\sphinxincludegraphics{{41eadb13e84a3aff677404210845642eae266f9eef2e54f251e1db9af63ccb4e}.png}

\end{sphinxuseclass}\end{sphinxVerbatimOutput}

\end{sphinxuseclass}
\end{sphinxuseclass}
\begin{DUlineblock}{0em}
\item[] \sphinxstylestrong{\large Wired\sphinxhyphen{}AND}
\end{DUlineblock}

\sphinxAtStartPar
Durch Wired\sphinxhyphen{}AND wird, sobald ein Teilnehmer LOW senden möchte, die Leitung auf LOW gehen
und alle anderen Teilnehmer überschreibt.
Dies ermöglicht die Arbitrierung
und das ein Teilnehmer die Clock Leitung auf LOW ziehen kann wenn dieser mehr Zeit braucht.

\begin{DUlineblock}{0em}
\item[] \sphinxstylestrong{\large Datanübertragunsrahmen}
\end{DUlineblock}

\begin{sphinxuseclass}{cell}
\begin{sphinxuseclass}{tag_hide-input}\begin{sphinxVerbatimOutput}

\begin{sphinxuseclass}{cell_output}
\noindent\sphinxincludegraphics{{0b6b92a2e72b18e72d591cfb019c05470b55a0e53e7d42a97a2578dad3b8755a}.png}

\end{sphinxuseclass}\end{sphinxVerbatimOutput}

\end{sphinxuseclass}
\end{sphinxuseclass}
\sphinxAtStartPar
\(A\)…Acknowledge SDA\sphinxhyphen{}LOW\\
\(\overline{A}\)…not Acknowledge\sphinxhyphen{}HIGH\\
\(S\)…Start Bedingung\\
\(P\)…Stopp Bedingung

\begin{DUlineblock}{0em}
\item[] \sphinxstylestrong{\large Ablauf Datensenden und Datenempfang}
\end{DUlineblock}

\sphinxAtStartPar
Beim Start wird bei SCL HIGH SDA auf LOW gezogen.

\sphinxAtStartPar
Zuerst findet die Arbitrierung statt,
jeder Master der Senden möchte Sendet die Slave\sphinxhyphen{}Adresse

\sphinxAtStartPar
Datensendung MSB zu LSB (Big\sphinxhyphen{}Endian)

\sphinxAtStartPar
Ein Block an Daten ist immer 8\sphinxhyphen{}Bit (1\sphinxhyphen{}Byte),\\
die Anzahl an Bytes ist Theoretisch unbegrenzt,
es kann sein das die Bausteine eine begrenzte Anzahl an Bytes senden/empfangen können.

\sphinxAtStartPar
Am Ende wird die Übertragung mit einem Acknowledge abgeschlossen.
Dieses gibt an ob der Teilnehmer die Übertragung weiterführen oder abbrechen will.
\begin{itemize}
\item {} 
\sphinxAtStartPar
\(\text{ACK} = 0\Rightarrow \text{Senden weiterführen}\)

\item {} 
\sphinxAtStartPar
\(\text{ACK} = 1\Rightarrow \text{Senden abbrechen}\)

\end{itemize}

\sphinxAtStartPar
Es liefert immer der Teilnehmer das ACK\sphinxhyphen{}Bit,
welche die Daten Empfängt (Master wenn Slave sendet, Slave wenn Master sendet)

\sphinxAtStartPar
Am Ende kommt die Stopp Bedingung,
wo bei HIGH SCL SDA auf HIGH gebracht wird.\\
Alternativ kann der Master die Repeated Start Bedingung senden,
womit er wieder die Slave Adresse aussendet und

\begin{sphinxuseclass}{cell}
\begin{sphinxuseclass}{tag_hide-input}\begin{sphinxVerbatimOutput}

\begin{sphinxuseclass}{cell_output}
\noindent\sphinxincludegraphics{{074477c49cf24d1eb072e1bd323f90e827d937b01818cf4743c2355e1522f780}.png}

\end{sphinxuseclass}\end{sphinxVerbatimOutput}

\end{sphinxuseclass}
\end{sphinxuseclass}
\sphinxAtStartPar
\(RS\)…Repated Start

\begin{DUlineblock}{0em}
\item[] \sphinxstylestrong{\large Busarbitrierung}
\end{DUlineblock}

\sphinxAtStartPar
Jeder sendet die Slave Adresse aus,
zu die er senden möchte.\\
Der jedes Mal wenn ein Teilnehmer überschrieben wird,
schaltet sich dieser Teilnehmer weg
und sendet den rezessiven Zustand (HIGH)

\begin{sphinxuseclass}{cell}
\begin{sphinxuseclass}{tag_hide-input}\begin{sphinxVerbatimOutput}

\begin{sphinxuseclass}{cell_output}
\noindent\sphinxincludegraphics{{58df0f92c4b94378375d658eda9ea5c6f409b2eed48bc7c009d8529b5fd02e50}.png}

\end{sphinxuseclass}\end{sphinxVerbatimOutput}

\end{sphinxuseclass}
\end{sphinxuseclass}
\sphinxAtStartPar
T2 Schaltet sich beim 3. Bit weg\\
T3 Schaltet sich beim 4. Bit weg

\begin{DUlineblock}{0em}
\item[] \sphinxstylestrong{\large Can}
\end{DUlineblock}

\begin{DUlineblock}{0em}
\item[] \sphinxstylestrong{\large Spezifikation}
\end{DUlineblock}

\sphinxAtStartPar
Controller Area Network

\sphinxAtStartPar
\sphinxstylestrong{Merkmale}
\begin{itemize}
\item {} 
\sphinxAtStartPar
Asynchron und Seriell

\item {} 
\sphinxAtStartPar
HW Realisierung

\item {} 
\sphinxAtStartPar
Ziel: Kabelbäume in Autos Reduzieren

\end{itemize}

\sphinxAtStartPar
\sphinxstylestrong{Netzwerk}
\begin{itemize}
\item {} 
\sphinxAtStartPar
CAN\sphinxhyphen{}Knoten

\item {} 
\sphinxAtStartPar
Linien\sphinxhyphen{}/Sterntopologie

\item {} 
\sphinxAtStartPar
verdrillte ungeschirmte Zweidrahtleitung

\item {} 
\sphinxAtStartPar
symmetrische Signalübertragung

\item {} 
\sphinxAtStartPar
Datenrate: max 1MBit/s bei 40m

\item {} 
\sphinxAtStartPar
\(120\Omega\) Terminierung

\item {} 
\sphinxAtStartPar
32 Teilnehmer pro Busstrang
mehrere durch Repeater

\item {} 
\sphinxAtStartPar
Bestandteile eines CAN\sphinxhyphen{}Knotens
\begin{itemize}
\item {} 
\sphinxAtStartPar
Host\\
Übergeordneter Host

\item {} 
\sphinxAtStartPar
CAN Controller\\
einheitliche Abwicklung des CAN Protokolls

\item {} 
\sphinxAtStartPar
CAN Transceiver\\
Ankopplung CAN\sphinxhyphen{}Controller an CAN\sphinxhyphen{}Bus

\end{itemize}

\item {} 
\sphinxAtStartPar
Differenzialübertragung

\item {} 
\sphinxAtStartPar
High\sphinxhyphen{} und Low\sphinxhyphen{}Speed Übertragung
\begin{itemize}
\item {} 
\sphinxAtStartPar
Highspeed\\
\(40\text{kBit/s}\)…\(1\text{MBit/s}\)
\begin{itemize}
\item {} 
\sphinxAtStartPar
HIGH: rezessiv, \(2.5V\) auf beiden Leitungen

\item {} 
\sphinxAtStartPar
LOW: dominant, CAN\(_{high} = 3.5V\) CAN\(_{low} = 1.5V\) \(U_{diff} = 2V\)

\end{itemize}

\item {} 
\sphinxAtStartPar
Lowspeed\\
\(5\text{kBit/s}\)…\(125\text{kBit/s}\)\\
CAN bleibt nur mit einer Plus Leitung funktionsfähig
\begin{itemize}
\item {} 
\sphinxAtStartPar
HIGH: rezessiv, CAN\(_{high} = 0V\) CAN\(_{low} = 5V\) \$U\_\{diff\} = 5V

\item {} 
\sphinxAtStartPar
LOW: dominant, CAN\(_{high} = 3.6V\) CAN\(_{low} = 1.6V\) \(U_{diff} = 2V\)

\end{itemize}

\end{itemize}

\item {} 
\sphinxAtStartPar
Dominanter und rezessiver Pegel

\item {} 
\sphinxAtStartPar
Data Frames nicht von Zeit sonder vom Auftreten spezieller Ereignisse

\item {} 
\sphinxAtStartPar
Nachrichtenlänge max 130 Bit

\item {} 
\sphinxAtStartPar
kein Zeitplan, Nachrichten werden versendet wenn sie anfallen \(\rightarrow\) Kollisionsgefahr

\item {} 
\sphinxAtStartPar
CSMA/CA

\item {} 
\sphinxAtStartPar
Verzögerung niederprioren Nachrichten \(\rightarrow\) Beeinträchtigung der Echtzeitfähigkeit

\end{itemize}

\begin{DUlineblock}{0em}
\item[] \sphinxstylestrong{\large Adressierung}
\end{DUlineblock}

\sphinxAtStartPar
Es wird die ID des Datagramms ausgesendet,
jeder der Interesse an diesen Datagramm hat wechselt in den Empfangsmodus und liest es ein

\begin{DUlineblock}{0em}
\item[] \sphinxstylestrong{\large Wired\sphinxhyphen{}AND}
\end{DUlineblock}

\sphinxAtStartPar
und a drittes mal\\
Leitung wird auf HIGH gezogen (rezessiver Pegel)
und sobald ein Teilnehmer LOW senden möchte wird LOW gesendet (dominanter Pegel).

\begin{DUlineblock}{0em}
\item[] \sphinxstylestrong{\large Datanübertragunsrahmen}
\end{DUlineblock}

\sphinxAtStartPar
\sphinxincludegraphics{{can_df}.png}
\begin{itemize}
\item {} 
\sphinxAtStartPar
Bus\sphinxhyphen{}Idle:\\
Ruhephase des Systems

\item {} 
\sphinxAtStartPar
SOF:\\
\sphinxstylestrong{S}tart \sphinxstylestrong{o}f \sphinxstylestrong{F}rame

\item {} 
\sphinxAtStartPar
Identifier:\\
ID der Botschaft für Arbitrierung

\item {} 
\sphinxAtStartPar
RTR:\\
\sphinxstylestrong{r}emote \sphinxstylestrong{t}ransmission \sphinxstylestrong{r}equest\\
kennzeichnet ob das Frame Daten enthält oder zum senden von Daten auffordert

\item {} 
\sphinxAtStartPar
IDE:\\
Identifier extension\\
Standardtelegramm low

\item {} 
\sphinxAtStartPar
DLC:\\
Data length Control\\
Längeninformationen über das Datenfeld

\item {} 
\sphinxAtStartPar
Data field:\\
enthält die Nutzdaten

\item {} 
\sphinxAtStartPar
CRC\sphinxhyphen{}Checksum:\\
CRC Checksumme

\item {} 
\sphinxAtStartPar
DEL:\\
CRC delimiter

\item {} 
\sphinxAtStartPar
ACK:\\
Alle Teilnehmer welche die Botschaft korrekt empfangen haben quittieren durch senden dominanten Pegel\\
Sender sendet rezessiven Pegel und erwartet überschrieben zu werden

\item {} 
\sphinxAtStartPar
DEL:\\
ACK delimiter

\item {} 
\sphinxAtStartPar
EOF:\\
kennzeichnet End of Frame\\
Bewusste Codierungs\sphinxhyphen{}Verletzung durch senden von mehr als 5 rezessive Bits

\item {} 
\sphinxAtStartPar
ITM:
Intermission, trennt Botschaften ab

\end{itemize}

\begin{DUlineblock}{0em}
\item[] \sphinxstylestrong{\large Ablauf Datensenden und Datenempfang}
\end{DUlineblock}
\begin{itemize}
\item {} 
\sphinxAtStartPar
Datenübertragung erfolgt mittels Nachrichtenrahmen \sphinxstylestrong{CAN Data Frames}

\item {} 
\sphinxAtStartPar
Nutzdaten bis acht Byte können in einem Frame übertragen werden

\item {} 
\sphinxAtStartPar
Jeder Data Frame steht jedem Knoten zur Übernahme zur Verfügung

\item {} 
\sphinxAtStartPar
Jeder Data Frame hat einen \sphinxstylestrong{Identifier (ID)}, welcher die Nachricht kennzeichnet

\end{itemize}

\begin{DUlineblock}{0em}
\item[] \sphinxstylestrong{\large Busarbitrierung}
\end{DUlineblock}
\begin{itemize}
\item {} 
\sphinxAtStartPar
CSMA/CA

\item {} 
\sphinxAtStartPar
verhindert Kollisionen

\item {} 
\sphinxAtStartPar
Identifier der CAN Botschaft zur Arbitrierung bitweise vom MSB zum LSB (Big\sphinxhyphen{}Endian)

\item {} 
\sphinxAtStartPar
CAN Botschaft mit der niedrigsten ID wird gewinnt und wird übertragen

\item {} 
\sphinxAtStartPar
Knoten welche die Arbitrierung verlieren gehen in Empfangsmodus und Warten bis Bus wieder frei

\end{itemize}

\sphinxAtStartPar
\sphinxincludegraphics{{can_arbit}.png}

\begin{DUlineblock}{0em}
\item[] \sphinxstylestrong{\large Physikalische und Strukturelle Fehlererkennungsmaßnahmen}
\end{DUlineblock}

\begin{DUlineblock}{0em}
\item[] \sphinxstylestrong{\large Bit\sphinxhyphen{}Stuffing}
\end{DUlineblock}

\sphinxAtStartPar
Bei Asynchroner Datenübertragung werden die Flanken zum synchronisieren des Datagramms verwendet.

\sphinxAtStartPar
Wenn allerdings länger der selbe Pegel bleibt,
kann dies zu Problemen bei der Auslesung führen.\\
Um dies zu verhindern,
überträgt man ein Stuffing Bit,
welches für eine Flanke sorgt.

\sphinxAtStartPar
Da beide Teilnehmer das Stuffing erwarten gibt es keine Probleme bei der Dekodierung

\begin{DUlineblock}{0em}
\item[] \sphinxstylestrong{\large OSI\sphinxhyphen{}ISO\sphinxhyphen{}Modell für Schnittstellen}
\end{DUlineblock}

\sphinxAtStartPar
\sphinxincludegraphics{{osii}.png}

\sphinxAtStartPar
Merksatz: “\sphinxstylestrong{\large P}lease \sphinxstylestrong{d}o \sphinxstylestrong{n}ot \sphinxstylestrong{t}hrow \sphinxstylestrong{s}alami \sphinxstylestrong{p}izzas \sphinxstylestrong{a}way!”

\sphinxAtStartPar
Als für einen Feldbus müssen mindestens 3 Schichten vorhanden sein:
\begin{itemize}
\item {} 
\sphinxAtStartPar
Application Layer\\
Zugriff auf das Kommunikationssystem (Software Library)

\item {} 
\sphinxAtStartPar
Data Link Layer\\
Datagramm Aufbau, Fehlererkennung, Buszugriff

\item {} 
\sphinxAtStartPar
Physical Layer\\
Wie werden die Bit übertragen

\end{itemize}

\sphinxAtStartPar
Optional:
\begin{itemize}
\item {} 
\sphinxAtStartPar
Network Layer\\
Übertragungsweg über Busknoten

\end{itemize}

\begin{DUlineblock}{0em}
\item[] \sphinxstylestrong{\large Regelungstechnik}
\end{DUlineblock}

\begin{sphinxuseclass}{cell}
\begin{sphinxuseclass}{tag_hide-input}
\end{sphinxuseclass}
\end{sphinxuseclass}
\begin{DUlineblock}{0em}
\item[] \sphinxstylestrong{\large Regelkreis}
\end{DUlineblock}

\begin{DUlineblock}{0em}
\item[] \sphinxstylestrong{\large Standardregelkreis}
\end{DUlineblock}

\begin{sphinxuseclass}{cell}
\begin{sphinxuseclass}{tag_hide-input}\begin{sphinxVerbatimOutput}

\begin{sphinxuseclass}{cell_output}
\noindent\sphinxincludegraphics{{ef7557c86738c52b98d8a5b5c9f65534c3ed703c31882b52336ac7f656b688e9}.png}

\end{sphinxuseclass}\end{sphinxVerbatimOutput}

\end{sphinxuseclass}
\end{sphinxuseclass}
\begin{DUlineblock}{0em}
\item[] \sphinxstylestrong{\large Blockschaltbild}
\end{DUlineblock}

\sphinxAtStartPar
Das Blockschaltbild ist die Darstellung einer Regelung in Funktionsblöcken.

\sphinxAtStartPar
Darstellung ohne Räumliche Zuordnung.
Mehrere Blöcke oft durch einzelnes wirkliches Element

\begin{DUlineblock}{0em}
\item[] \sphinxstylestrong{\large Bestimmung Sprungantwort}
\end{DUlineblock}

\sphinxAtStartPar
Übertragungsfunktion ist \(G(s) = \frac{U_a(s)}{U_e(s)}\)

\sphinxAtStartPar
Bei Sprungantwort \(U_e = \frac{1}{s}\)\\
Einsetzen und auf \(U_a(s)\) umformen,
wenn möglich vereinfachen
und mit Laplace Tabelle zurück transformieren.

\begin{DUlineblock}{0em}
\item[] \sphinxstylestrong{\large Ü\sphinxhyphen{}Funktion zwischen Ausgängen und Eingängen von Regelkreisen}
\end{DUlineblock}
\begin{itemize}
\item {} 
\sphinxAtStartPar
Ein Pfeil bedeutet Multiplikation, die Werte sind dabei die Werte in den Boxen

\item {} 
\sphinxAtStartPar
Ein Kreis (oft mit Vorzeichen) bezeichnet eine Addition/Subtraktion

\end{itemize}

\sphinxAtStartPar
Es gelten die selben Arithmetischen Regeln wie Regulär in Mathe.

\sphinxAtStartPar
Die Übertragungsfunktion gibt immer \(\frac{U_a(s)}{U_e(s)}\) an.

\begin{DUlineblock}{0em}
\item[] \sphinxstylestrong{\large Beschreibung im Zeit\sphinxhyphen{} und Frequenzbereich}
\end{DUlineblock}

\sphinxAtStartPar
Im Frequenzbereich/Bildbereich beschreibt das Bode Diagramm (Amplitudengang \& Phasengang).

\sphinxAtStartPar
Im Zeitbereich beschreibt die Sprungantwort das Element.

\begin{DUlineblock}{0em}
\item[] \sphinxstylestrong{\large Rückwirkungsfreiheit}
\end{DUlineblock}

\sphinxAtStartPar
Damit zwei Boxen wirklich multipliziert werden können, müssen diese Rückwirkungsfrei sein,
sprich die zweite Box darf die erste nicht Beeinflussen.
Ansonsten müssen diese Aufwendig, über die Schaltungen, aneinander geschaltet werden.

\sphinxAtStartPar
Die Rückwirkungsfreiheit wird über OPVs erreicht,
welche die Schaltungen entkoppeln
\begin{itemize}
\item {} 
\sphinxAtStartPar
Zwei RC\sphinxhyphen{}TPs
\begin{itemize}
\item {} 
\sphinxAtStartPar
\(G(s)\) wäre \(\frac{1}{1+sR_1C_1}\cdot\frac{1}{1+sR_2C_2}\)

\item {} 
\sphinxAtStartPar
Kann nicht sein

\end{itemize}

\item {} 
\sphinxAtStartPar
Blöcke können als Eingang und Ausgang modelliert werden
\begin{itemize}
\item {} 
\sphinxAtStartPar
für Rückwirkungsfreiheit
\begin{itemize}
\item {} 
\sphinxAtStartPar
\(r_a << r_e\)

\item {} 
\sphinxAtStartPar
(Strenggenommen: \(|z_a| < |z_e|\) )

\item {} 
\sphinxAtStartPar
Bedeutung: Der Ausgangswiderstand muss im Verhältnis zum Eingangswiderstand vernachlässigbar sein

\end{itemize}

\end{itemize}

\item {} 
\sphinxAtStartPar
für zwei TPs würde das heißen:
\begin{itemize}
\item {} 
\sphinxAtStartPar
\(z_a = R_1 || C_1\)

\item {} 
\sphinxAtStartPar
\(z_e = R_2 || C_2\)

\item {} 
\sphinxAtStartPar
meist einfacher mit OPVs

\end{itemize}

\end{itemize}

\begin{DUlineblock}{0em}
\item[] \sphinxstylestrong{\large Übertragungsfunktion}
\end{DUlineblock}

\sphinxAtStartPar
Die Übertragungsfunktion stellt das Verhältnis von Ausgang zu Eingang dar \(\frac{U_a}{U_e}\)

\sphinxAtStartPar
Es gibt sowohl die Übertragungsfunktion für den Zeit und den Bildbereich,
wobei im meist die im Bildbereich relevant ist um die Amplitude und Phase im Verhältnis zur Eingangsfrequenz darzustellen.
\(\frac{U_a(s)}{U_e{s}} = G(s)\)\\
Die Übertragungsfunktion im Bildbereich kann auch für die Sprungantwort verwendet werden \(\left(U_e = \frac{1}{s}\right)\)

\begin{DUlineblock}{0em}
\item[] \sphinxstylestrong{\large Laplace\sphinxhyphen{}Transformation}
\end{DUlineblock}

\begin{DUlineblock}{0em}
\item[] \sphinxstylestrong{\large Vorgehensweise bei Systemantwort (Sprungantwort)}
\end{DUlineblock}

\sphinxAtStartPar
\(G(s) = \frac{U_a}{U_e} \Rightarrow U_a(s) = G(s) \cdot U_e \wedge U_e = \frac{1}{s} \Rightarrow U_a(s) = \frac{G(s)}{s} \)

\sphinxAtStartPar
\(\mathscr{L}^{-1}\left\{U_a(s)\right\} = \mathscr{L}^{-1}\left\{\frac{G(s)}{s}\right\}\) = …

\begin{DUlineblock}{0em}
\item[] \sphinxstylestrong{\large Anwendung von AWT, EWT}
\end{DUlineblock}

\sphinxAtStartPar
\(\lim_{s\rightarrow 0+}   ~s \cdot F(s) = \lim_{t\rightarrow\infty}f(t)\)

\sphinxAtStartPar
\(\lim_{s\rightarrow\infty}~s \cdot F(s) = \lim_{s\rightarrow 0+}   f(t)\)

\begin{DUlineblock}{0em}
\item[] \sphinxstylestrong{Partialbruchzerlegung für \(\mathscr{L}^{-1}\)}
\end{DUlineblock}

\sphinxAtStartPar
Geogebra Befehl:
\begin{quote}

\sphinxAtStartPar
PartialFractions(\(Function\))
\end{quote}

\sphinxAtStartPar
macht die Zerlegung automatisch,
dies wird verwendet um Komplexe Brüche mit Polynomen auf einfachere Brüche aufzuteilen.\\
Welche dann \sphinxhyphen{} im Idealfall \sphinxhyphen{}  in der Transformations\sphinxhyphen{}Tabelle vorhanden sind.

\begin{DUlineblock}{0em}
\item[] \sphinxstylestrong{\large Inverse Laplace\sphinxhyphen{}Transformation mittels Transformations\sphinxhyphen{}Tabelle}
\end{DUlineblock}
\begin{itemize}
\item {} 
\sphinxAtStartPar
Man nimmt die jeweilige Form der Funktion im Zeit/Bildbereich

\item {} 
\sphinxAtStartPar
Man bringt die Funktion in eine Form, welche in der Tabelle gegeben ist
!! Form muss vorhanden sein
Partial\sphinxhyphen{}Bruch bei Polynombrüchen

\item {} 
\sphinxAtStartPar
Man suche die Werte für die variablen Konstanten in der Funktionen (e.g. \(a\))

\item {} 
\sphinxAtStartPar
Man nehme die Funktion auf der anderen Seite und setze die Werte wieder ein

\end{itemize}
\begin{itemize}
\item {} 
\sphinxAtStartPar
Enjoy

\end{itemize}

\sphinxAtStartPar
\sphinxincludegraphics{{laplace_table}.png}

\begin{DUlineblock}{0em}
\item[] \sphinxstylestrong{\large RT\sphinxhyphen{}Modelle der OPV\sphinxhyphen{}Grundschaltung}
\end{DUlineblock}

\begin{DUlineblock}{0em}
\item[] \sphinxstylestrong{\large Inv. und nicht inv OPV\sphinxhyphen{}Verstärker}
\end{DUlineblock}

\begin{sphinxuseclass}{cell}
\begin{sphinxuseclass}{tag_hide-input}\begin{sphinxVerbatimOutput}

\begin{sphinxuseclass}{cell_output}
\noindent\sphinxincludegraphics{{3ef500ba50c8102bd0bb092818d358ddc0e33338b716f0dd8b8d067a80881ac5}.png}

\end{sphinxuseclass}\end{sphinxVerbatimOutput}

\end{sphinxuseclass}
\end{sphinxuseclass}
\begin{sphinxuseclass}{cell}\begin{sphinxVerbatimInput}

\begin{sphinxuseclass}{cell_input}
\begin{sphinxVerbatim}[commandchars=\\\{\}]
\PYG{k}{with} \PYG{n}{schemdraw}\PYG{o}{.}\PYG{n}{Drawing}\PYG{p}{(}\PYG{p}{)} \PYG{k}{as} \PYG{n}{d}\PYG{p}{:}
    \PYG{n}{d} \PYG{o}{+}\PYG{o}{=} \PYG{n}{elm}\PYG{o}{.}\PYG{n}{Line}\PYG{p}{(}\PYG{p}{)}\PYG{o}{.}\PYG{n}{idot}\PYG{p}{(}\PYG{n+nb}{open}\PYG{o}{=}\PYG{k+kc}{True}\PYG{p}{)}
    \PYG{n}{d} \PYG{o}{+}\PYG{o}{=} \PYG{p}{(}\PYG{n}{amp} \PYG{o}{:=} \PYG{n}{elm}\PYG{o}{.}\PYG{n}{Opamp}\PYG{p}{(}\PYG{p}{)}\PYG{o}{.}\PYG{n}{anchor}\PYG{p}{(}\PYG{l+s+s1}{\PYGZsq{}}\PYG{l+s+s1}{in2}\PYG{l+s+s1}{\PYGZsq{}}\PYG{p}{)}\PYG{o}{.}\PYG{n}{drop}\PYG{p}{(}\PYG{l+s+s1}{\PYGZsq{}}\PYG{l+s+s1}{out}\PYG{l+s+s1}{\PYGZsq{}}\PYG{p}{)}\PYG{o}{.}\PYG{n}{flip}\PYG{p}{(}\PYG{p}{)}\PYG{p}{)}
    \PYG{n}{d} \PYG{o}{+}\PYG{o}{=} \PYG{n}{elm}\PYG{o}{.}\PYG{n}{Line}\PYG{p}{(}\PYG{p}{)}\PYG{o}{.}\PYG{n}{length}\PYG{p}{(}\PYG{l+m+mi}{1}\PYG{p}{)}\PYG{o}{.}\PYG{n}{dot}\PYG{p}{(}\PYG{p}{)}
    \PYG{n}{d}\PYG{o}{.}\PYG{n}{push}\PYG{p}{(}\PYG{p}{)}

    \PYG{n}{d} \PYG{o}{+}\PYG{o}{=} \PYG{n}{elm}\PYG{o}{.}\PYG{n}{Line}\PYG{p}{(}\PYG{p}{)}\PYG{o}{.}\PYG{n}{length}\PYG{p}{(}\PYG{l+m+mi}{1}\PYG{p}{)}\PYG{o}{.}\PYG{n}{dot}\PYG{p}{(}\PYG{n+nb}{open}\PYG{o}{=}\PYG{k+kc}{True}\PYG{p}{)}
    \PYG{n}{d}\PYG{o}{.}\PYG{n}{pop}\PYG{p}{(}\PYG{p}{)}
    \PYG{n}{d} \PYG{o}{+}\PYG{o}{=} \PYG{n}{elm}\PYG{o}{.}\PYG{n}{ResistorIEC}\PYG{p}{(}\PYG{p}{)}\PYG{o}{.}\PYG{n}{down}\PYG{p}{(}\PYG{p}{)}\PYG{o}{.}\PYG{n}{dot}\PYG{p}{(}\PYG{p}{)}\PYG{o}{.}\PYG{n}{label}\PYG{p}{(}\PYG{l+s+s1}{\PYGZsq{}}\PYG{l+s+s1}{\PYGZdl{}R\PYGZus{}1\PYGZdl{}}\PYG{l+s+s1}{\PYGZsq{}}\PYG{p}{)}
    \PYG{n}{d}\PYG{o}{.}\PYG{n}{push}\PYG{p}{(}\PYG{p}{)}
    \PYG{n}{d} \PYG{o}{+}\PYG{o}{=} \PYG{n}{elm}\PYG{o}{.}\PYG{n}{Line}\PYG{p}{(}\PYG{p}{)}\PYG{o}{.}\PYG{n}{left}\PYG{p}{(}\PYG{p}{)}\PYG{o}{.}\PYG{n}{length}\PYG{p}{(}\PYG{l+m+mi}{5}\PYG{p}{)}
    \PYG{n}{d} \PYG{o}{+}\PYG{o}{=} \PYG{n}{elm}\PYG{o}{.}\PYG{n}{Wire}\PYG{p}{(}\PYG{l+s+s1}{\PYGZsq{}}\PYG{l+s+s1}{|\PYGZhy{}}\PYG{l+s+s1}{\PYGZsq{}}\PYG{p}{)}\PYG{o}{.}\PYG{n}{to}\PYG{p}{(}\PYG{n}{amp}\PYG{o}{.}\PYG{n}{in1}\PYG{p}{)}
    \PYG{n}{d}\PYG{o}{.}\PYG{n}{pop}\PYG{p}{(}\PYG{p}{)}
    \PYG{n}{d} \PYG{o}{+}\PYG{o}{=} \PYG{n}{elm}\PYG{o}{.}\PYG{n}{ResistorIEC}\PYG{p}{(}\PYG{p}{)}\PYG{o}{.}\PYG{n}{down}\PYG{p}{(}\PYG{p}{)}\PYG{o}{.}\PYG{n}{label}\PYG{p}{(}\PYG{l+s+s1}{\PYGZsq{}}\PYG{l+s+s1}{\PYGZdl{}R\PYGZus{}2\PYGZdl{}}\PYG{l+s+s1}{\PYGZsq{}}\PYG{p}{)}
    \PYG{n}{d} \PYG{o}{+}\PYG{o}{=} \PYG{n}{elm}\PYG{o}{.}\PYG{n}{Ground}\PYG{p}{(}\PYG{p}{)}
\end{sphinxVerbatim}

\end{sphinxuseclass}\end{sphinxVerbatimInput}
\begin{sphinxVerbatimOutput}

\begin{sphinxuseclass}{cell_output}
\noindent\sphinxincludegraphics{{83f15d631ea384cbe546e4c873434cf4672e582f0af0ddf05f2a943621c8038e}.png}

\end{sphinxuseclass}\end{sphinxVerbatimOutput}

\end{sphinxuseclass}
\sphinxAtStartPar
\(U_a = U_e\cdot \left(1 + \frac{R_2}{R_1}\right)\)

\begin{DUlineblock}{0em}
\item[] \sphinxstylestrong{\large Bestimmung der OPV\sphinxhyphen{}Verstärkung}
\end{DUlineblock}

\begin{sphinxuseclass}{cell}
\begin{sphinxuseclass}{tag_hide-input}\begin{sphinxVerbatimOutput}

\begin{sphinxuseclass}{cell_output}
\noindent\sphinxincludegraphics{{803526d16ad56d7d0ca2ec1848cd50aa4a1e67645c1cadbd6503bc7b1519f52c}.png}

\end{sphinxuseclass}\end{sphinxVerbatimOutput}

\end{sphinxuseclass}
\end{sphinxuseclass}
\sphinxAtStartPar
\(U_a = U_e\cdot -\frac{Z_2}{Z_1}\)

\begin{DUlineblock}{0em}
\item[] \sphinxstylestrong{\large Auswirkung auf die Regelabweichung}
\end{DUlineblock}

\begin{DUlineblock}{0em}
\item[] \sphinxstylestrong{\large Zusammengesetzte Frequenzgänge}
\end{DUlineblock}

\begin{DUlineblock}{0em}
\item[] \sphinxstylestrong{\large Knickzug vom Amplituden\sphinxhyphen{} und Phasengang}
\end{DUlineblock}

\sphinxAtStartPar
Bei der Knickkreisfrequenz \(\omega_g\) ist bei den meisten Elementen ein Umschwung der Verstärkerfunktion, bspw. von \(0dB/dek\) auf \(-20dB/dek\) bei einem PT1\sphinxhyphen{}Element

\begin{DUlineblock}{0em}
\item[] \sphinxstylestrong{\large Zerlegung von Übertragungsfunktionen in Grundglieder}
\end{DUlineblock}

\sphinxAtStartPar
Um Übertragungsfunktion muss in einzelne Elemente zu zerlegen muss es zu einer Reihe an Multiplikationen von \(T_1\cdot s\), \(\frac{1}{T_1\cdot s}\), \(\frac{k}{1+T_1\cdot s}\), \(k\cdot(1 + T_1\cdot s)\) zerlegt werden.

\begin{DUlineblock}{0em}
\item[] \sphinxstylestrong{Rekonstruktion von \(G(s)\) aus Amplitudenverlauf}
\end{DUlineblock}

\sphinxAtStartPar
Zum Rekonstruieren müssen bei einer Änderung jeweilige Elemente (je nach Veränderung; meist PT1 oder PD) dazugeschaltet werden,
bei den zusätzlichen Elementen muss die Verstärkung dimensioniert werden (bei allen außer dem ersten meist \(1\)),
und die Kreisfrequenz des Knicks dimensioniert werden.
Am Anfang muss das jeweilige Element ausgewählt werden welches die Funktion vor jedem Knick gut beschreibt

\sphinxAtStartPar
Um auf die gesamt OPV\sphinxhyphen{}Schaltung zu kommen müssen die OPV\sphinxhyphen{}Schaltungen der einzelnen Elemente aneinander gehängt werden (Vorzeichen beachten)

\begin{DUlineblock}{0em}
\item[] \sphinxstylestrong{\large Grundglieder}
\end{DUlineblock}

\begin{DUlineblock}{0em}
\item[] \sphinxstylestrong{\large I}
\end{DUlineblock}

\begin{DUlineblock}{0em}
\item[] \sphinxstylestrong{\large Übertragungsfunktion}
\end{DUlineblock}

\sphinxAtStartPar
\(G(s) = \frac{1}{s}\)

\begin{DUlineblock}{0em}
\item[] \sphinxstylestrong{\large Sprungantwort}
\end{DUlineblock}

\sphinxAtStartPar
\(U_a(s) = \frac{1}{s} \cdot U_e = \frac{1}{s}\cdot \frac{1}{s} = \frac{1}{s^2}\)

\sphinxAtStartPar
\(u_a(t) = t\)

\begin{DUlineblock}{0em}
\item[] \sphinxstylestrong{\large Bode\sphinxhyphen{}Diagramm}
\end{DUlineblock}

\sphinxAtStartPar
\(G(i\omega) = \frac{1}{T_1\cdot i\omega}\)

\begin{DUlineblock}{0em}
\item[] \sphinxstylestrong{\large Amplitudengang}
\end{DUlineblock}

\sphinxAtStartPar
\(|G(s)| = \frac{1}{\omega}\)

\sphinxAtStartPar
\(log|G(s)| = - 20\cdot log(s)\)

\begin{DUlineblock}{0em}
\item[] \sphinxstylestrong{\large Phasengang}
\end{DUlineblock}

\sphinxAtStartPar
\(arg(G(s)) = 0 - \frac{\pi}{2} = 0- 90°\)

\begin{sphinxuseclass}{cell}
\begin{sphinxuseclass}{tag_hide-input}\begin{sphinxVerbatimOutput}

\begin{sphinxuseclass}{cell_output}
\begin{sphinxVerbatim}[commandchars=\\\{\}]
\PYGZsq{}\PYGZsq{}
\end{sphinxVerbatim}

\noindent\sphinxincludegraphics{{36d6a5a8506b06baf21d12ae0d7911b2a5e7c5cdae2cfcb411afc92af8e68a2c}.png}

\end{sphinxuseclass}\end{sphinxVerbatimOutput}

\end{sphinxuseclass}
\end{sphinxuseclass}
\begin{DUlineblock}{0em}
\item[] \sphinxstylestrong{\large OPV\sphinxhyphen{}Schaltung}
\end{DUlineblock}

\begin{sphinxuseclass}{cell}
\begin{sphinxuseclass}{tag_hide-input}\begin{sphinxVerbatimOutput}

\begin{sphinxuseclass}{cell_output}
\noindent\sphinxincludegraphics{{ad15d477245533023cd4fde51bed34c562b48d7c88876938d8ab865fb5ebe81c}.png}

\end{sphinxuseclass}\end{sphinxVerbatimOutput}

\end{sphinxuseclass}
\end{sphinxuseclass}
\begin{DUlineblock}{0em}
\item[] \sphinxstylestrong{P}
\end{DUlineblock}

\begin{DUlineblock}{0em}
\item[] \sphinxstylestrong{\large Übertragungsfunktion}
\end{DUlineblock}

\sphinxAtStartPar
\(G(s) = k\)

\begin{DUlineblock}{0em}
\item[] \sphinxstylestrong{\large Sprungantwort}
\end{DUlineblock}

\sphinxAtStartPar
\(U_a(s) = k\cdot\frac{1}{s}\)

\sphinxAtStartPar
\(u_a(t) = k\cdot \sigma(t)\)

\begin{DUlineblock}{0em}
\item[] \sphinxstylestrong{\large Bodediagramm}
\end{DUlineblock}

\sphinxAtStartPar
\(G(i\omega) = k\)

\begin{DUlineblock}{0em}
\item[] \sphinxstylestrong{\large Amplitudengang}
\end{DUlineblock}

\sphinxAtStartPar
\(|G(s)| = k\)

\sphinxAtStartPar
\(log|G(s)| = - 20\cdot log(k)\)

\begin{DUlineblock}{0em}
\item[] \sphinxstylestrong{\large Phasengang}
\end{DUlineblock}

\sphinxAtStartPar
\(arg(G(s)) = 0\)

\begin{sphinxuseclass}{cell}
\begin{sphinxuseclass}{tag_hide-input}\begin{sphinxVerbatimOutput}

\begin{sphinxuseclass}{cell_output}
\begin{sphinxVerbatim}[commandchars=\\\{\}]
[\PYGZlt{}matplotlib.axis.YTick at 0x2b4b91d8490\PYGZgt{},
 \PYGZlt{}matplotlib.axis.YTick at 0x2b4b91b3750\PYGZgt{},
 \PYGZlt{}matplotlib.axis.YTick at 0x2b4b9a72c50\PYGZgt{},
 \PYGZlt{}matplotlib.axis.YTick at 0x2b4b91dbc10\PYGZgt{},
 \PYGZlt{}matplotlib.axis.YTick at 0x2b4b9a67850\PYGZgt{}]
\end{sphinxVerbatim}

\noindent\sphinxincludegraphics{{dfe64c450e4bee536b941ef9a5fbd2dc6a1db179f9ba82230ebaf4cfad5692bd}.png}

\end{sphinxuseclass}\end{sphinxVerbatimOutput}

\end{sphinxuseclass}
\end{sphinxuseclass}
\begin{DUlineblock}{0em}
\item[] \sphinxstylestrong{\large OPV\sphinxhyphen{}Schaltung}
\end{DUlineblock}

\begin{sphinxuseclass}{cell}
\begin{sphinxuseclass}{tag_hide-input}\begin{sphinxVerbatimOutput}

\begin{sphinxuseclass}{cell_output}
\noindent\sphinxincludegraphics{{b189ef5d4f959ad1d96431a26ba5a616eadaf041c12aabacf04a1283802787a3}.png}

\end{sphinxuseclass}\end{sphinxVerbatimOutput}

\end{sphinxuseclass}
\end{sphinxuseclass}
\begin{DUlineblock}{0em}
\item[] \sphinxstylestrong{\large D}
\end{DUlineblock}

\begin{DUlineblock}{0em}
\item[] \sphinxstylestrong{\large Übertragungsfunktion}
\end{DUlineblock}

\sphinxAtStartPar
\(G(s) = T_1\cdot s\)

\begin{DUlineblock}{0em}
\item[] \sphinxstylestrong{\large Sprungantwort}
\end{DUlineblock}

\sphinxAtStartPar
\(U_a(s) = T_1\cdot s\cdot\frac{1}{s}\)

\sphinxAtStartPar
\(u_a(t) = k\cdot \delta(t)\)

\begin{DUlineblock}{0em}
\item[] \sphinxstylestrong{\large Bodediagramm}
\end{DUlineblock}

\sphinxAtStartPar
\(G(i\omega) = T_1 \cdot i\omega\)

\begin{DUlineblock}{0em}
\item[] \sphinxstylestrong{\large Amplitudengang}
\end{DUlineblock}

\sphinxAtStartPar
\(|G(s)| = T_1\cdot\omega\)

\sphinxAtStartPar
\(log|G(s)| = - 20\cdot log(T_1\cdot\omega)\)

\begin{DUlineblock}{0em}
\item[] \sphinxstylestrong{\large Phasengang}
\end{DUlineblock}

\sphinxAtStartPar
\(arg(G(s)) = 90°\)

\begin{sphinxuseclass}{cell}
\begin{sphinxuseclass}{tag_hide-input}\begin{sphinxVerbatimOutput}

\begin{sphinxuseclass}{cell_output}
\begin{sphinxVerbatim}[commandchars=\\\{\}]
[\PYGZlt{}matplotlib.axis.YTick at 0x2b4b9ae1790\PYGZgt{},
 \PYGZlt{}matplotlib.axis.YTick at 0x2b4b92ab6d0\PYGZgt{},
 \PYGZlt{}matplotlib.axis.YTick at 0x2b4b92fc4d0\PYGZgt{},
 \PYGZlt{}matplotlib.axis.YTick at 0x2b4ba1c88d0\PYGZgt{},
 \PYGZlt{}matplotlib.axis.YTick at 0x2b4b902fb10\PYGZgt{}]
\end{sphinxVerbatim}

\noindent\sphinxincludegraphics{{2fb2c8065f97865941dfc7f79ea05b0a593cf46dcf6be4990751d97571c21f34}.png}

\end{sphinxuseclass}\end{sphinxVerbatimOutput}

\end{sphinxuseclass}
\end{sphinxuseclass}
\begin{DUlineblock}{0em}
\item[] \sphinxstylestrong{\large OPV\sphinxhyphen{}Schaltung}
\end{DUlineblock}

\begin{sphinxuseclass}{cell}
\begin{sphinxuseclass}{tag_hide-input}\begin{sphinxVerbatimOutput}

\begin{sphinxuseclass}{cell_output}
\noindent\sphinxincludegraphics{{40a11d5e777836e422d6b0e6c76bf8e6285a085a91114969a5b94c033278c671}.png}

\end{sphinxuseclass}\end{sphinxVerbatimOutput}

\end{sphinxuseclass}
\end{sphinxuseclass}
\begin{DUlineblock}{0em}
\item[] \sphinxstylestrong{\large PT1, PI, PD, DT1 Zusammen}
\end{DUlineblock}


\begin{savenotes}\sphinxattablestart
\centering
\begin{tabulary}{\linewidth}[t]{|T|T|T|T|T|}
\hline

\sphinxAtStartPar

&\sphinxstyletheadfamily 
\sphinxAtStartPar
PT1
&\sphinxstyletheadfamily 
\sphinxAtStartPar
PD
&\sphinxstyletheadfamily 
\sphinxAtStartPar
PI
&\sphinxstyletheadfamily 
\sphinxAtStartPar
DT1
\\
\hline
\sphinxAtStartPar
\(G(s)\)
&
\sphinxAtStartPar
\(k\cdot\frac{1}{1 + T_1\cdot s}\)
&
\sphinxAtStartPar
\(k\cdot(1 + T_1\cdot s)\)
&
\sphinxAtStartPar
\(k\cdot\frac{T_1\cdot s + 1}{T_1 \cdot s}\)
&
\sphinxAtStartPar
\(k\cdot\frac{T_1\cdot s}{1+T_1\cdot s}\)
\\
\hline
\sphinxAtStartPar
\(|G(i\omega)|\)
&
\sphinxAtStartPar
\(|k|\cdot\frac{1}{\sqrt{1 + T_1^2\cdot\omega^2}}\)
&
\sphinxAtStartPar
\(k\cdot\sqrt{1 + T_1^2\cdot \omega^2}\)
&
\sphinxAtStartPar
\(|k|\cdot\frac{\sqrt{1 + T_1^2\omega^2}}{T_1\cdot \omega}\)
&
\sphinxAtStartPar
\(|k|\cdot\frac{T_1\cdot\omega}{\sqrt{1 + T_1^2\omega^2}}\)
\\
\hline
\sphinxAtStartPar
\(arg(G(i\omega)|\)
&
\sphinxAtStartPar
\(0 - arctan(T_1\omega)\)
&
\sphinxAtStartPar
\(arctan(T_1\cdot\omega)\)
&
\sphinxAtStartPar
\(arctan(T_1\cdot\omega)-90°\)
&
\sphinxAtStartPar
\(90°-arctan(T_1\cdot\omega)\)
\\
\hline
\end{tabulary}
\par
\sphinxattableend\end{savenotes}

\begin{DUlineblock}{0em}
\item[] \sphinxstylestrong{\large Amplitudengang für Name}
\end{DUlineblock}

\sphinxAtStartPar
Beim Verbinden von zwei Punkten durch den Mittelpunkt,
ergibt sich der Amplitudengang für das Regelelement für die dazugehörigen Buchstaben,
von dieser Form kann \(G(s)\) hergeleitet werden und von dieser der Rest

\begin{sphinxuseclass}{cell}
\begin{sphinxuseclass}{tag_hide-input}\begin{sphinxVerbatimOutput}

\begin{sphinxuseclass}{cell_output}
\begin{sphinxVerbatim}[commandchars=\\\{\}]
(\PYGZhy{}3.5, 3.5)
\end{sphinxVerbatim}

\noindent\sphinxincludegraphics{{4df0b41922b7ce5ce666ba2e4693fdb9434785af6adf2cccee950aa6ce713c43}.png}

\end{sphinxuseclass}\end{sphinxVerbatimOutput}

\end{sphinxuseclass}
\end{sphinxuseclass}
\begin{DUlineblock}{0em}
\item[] \sphinxstylestrong{\large Sprungantwort}
\end{DUlineblock}

\sphinxAtStartPar
Man nimmt \(G(s)\), \(G(s) = \frac{U_a(s)}{U_e(s)}\), wenn dies nun auf \(U_a(s)\) Umgeformt wird so ergibt sich \(U_a(s) = G(s)\cdot U_e(s)\).

\sphinxAtStartPar
Um die Sprungantwort zu berechnen wir in das Regelglied \(U_e(s) = \frac{1}{s}\) geschickt.
Dadurch ergibt sich: \(U_a(s) = G(s)\cdot\frac{1}{s}\).
Hier muss man nun für \(G(s)\), für das entsprechende Regelelement einsetzen und anschließend Rücktransformieren.

\begin{DUlineblock}{0em}
\item[] \sphinxstylestrong{\large OPV\sphinxhyphen{}Schaltung}
\end{DUlineblock}

\sphinxAtStartPar
Die Form des Amplitudenganges über den unteren Graphen legen und die nächste Vertikale Linie nehmen,
Wenn diese näher an Seriell ist so sind Kapazität und Widerstand in Serie geschaltet,
bei Parallel, Parallel.
Wenn der Linie zu Output geht so ist die Verschaltung von Widerstand und Kapazität am Ausgangspfad
und wenn näher bei Input beim Eingang.



\begin{sphinxuseclass}{cell}
\begin{sphinxuseclass}{tag_hide-input}\begin{sphinxVerbatimOutput}

\begin{sphinxuseclass}{cell_output}
\begin{sphinxVerbatim}[commandchars=\\\{\}]
(\PYGZhy{}2.2, 2.2, \PYGZhy{}3.3, 3.3)
\end{sphinxVerbatim}

\noindent\sphinxincludegraphics{{2ec5950e2df0787f28cc2707b62ec0ef6d79b6c1a64d8d409d536c8910ddee13}.png}

\end{sphinxuseclass}\end{sphinxVerbatimOutput}

\end{sphinxuseclass}
\end{sphinxuseclass}
\begin{DUlineblock}{0em}
\item[] \sphinxstylestrong{\large IT1}
\end{DUlineblock}

\sphinxAtStartPar
Zusammengesetzt aus einem I\sphinxhyphen{} und einem PT1\sphinxhyphen{}Regler.

\sphinxAtStartPar
Amplitudengänge addieren sich im logarithmischen Bereich.
Und Phasengänge Addieren sich.

\sphinxAtStartPar
\(G(s) = \frac{k}{T_I\cdot s\cdot(1 + T_1\cdot s)}\)
…

\begin{sphinxuseclass}{cell}
\begin{sphinxuseclass}{tag_hide-input}\begin{sphinxVerbatimOutput}

\begin{sphinxuseclass}{cell_output}
\noindent\sphinxincludegraphics{{a558b0f0a7873463fae80607e680460662f0329f24b38a378703da7c9cbbe22e}.png}

\end{sphinxuseclass}\end{sphinxVerbatimOutput}

\end{sphinxuseclass}
\end{sphinxuseclass}
\begin{sphinxuseclass}{cell}
\begin{sphinxuseclass}{tag_hide-input}\begin{sphinxVerbatimOutput}

\begin{sphinxuseclass}{cell_output}
\noindent\sphinxincludegraphics{{e02d93da84f8e38a261ab6596c626b48516a1696488ef079177f5f16575f3a41}.png}

\end{sphinxuseclass}\end{sphinxVerbatimOutput}

\end{sphinxuseclass}
\end{sphinxuseclass}
\begin{DUlineblock}{0em}
\item[] \sphinxstylestrong{\large PDT1}
\end{DUlineblock}

\sphinxAtStartPar
Das PDT1\sphinxhyphen{}Element setzt sich aus 1xPT1 und 1xPD \sphinxhyphen{}Element zusammen

\sphinxAtStartPar
\(G(s) = \frac{\left(1+sT_1\right)}{\left(1+sT_1\right)}\)

\sphinxAtStartPar
\(|G(j\omega)| = \frac{\sqrt{1+\omega^2T_1^2}}{\sqrt{1 + \omega^2T_2^2}}\)

\sphinxAtStartPar
\(arg(j\omega) = atan(\omega T_1) - atan(\omega T_2)\)

\begin{sphinxuseclass}{cell}
\begin{sphinxuseclass}{tag_hide-input}\begin{sphinxVerbatimOutput}

\begin{sphinxuseclass}{cell_output}
\begin{sphinxVerbatim}[commandchars=\\\{\}]
[\PYGZlt{}matplotlib.axis.YTick at 0x2b4bb1241d0\PYGZgt{},
 \PYGZlt{}matplotlib.axis.YTick at 0x2b4bb0f36d0\PYGZgt{}]
\end{sphinxVerbatim}

\noindent\sphinxincludegraphics{{fa5c11239fe7d5dfd859d725cf0b2672209fc775e59ce6e34eab01902532ef4f}.png}

\end{sphinxuseclass}\end{sphinxVerbatimOutput}

\end{sphinxuseclass}
\end{sphinxuseclass}
\sphinxAtStartPar
OPV\sphinxhyphen{}Schaltung ist ein inv Verstärker mit \(R||C\) für \(Z_1\) \sphinxstylestrong{und} \(Z_2\)

\begin{DUlineblock}{0em}
\item[] \sphinxstylestrong{\large Sprungantwort}
\end{DUlineblock}

\sphinxAtStartPar
\(U_a(s) = \frac{1 + sT_1}{1 + sT_2} \cdot \frac{1}{s}\)

\sphinxAtStartPar
\(= \frac{1 + sT_1}{s + s^2T_2} \)\\
mit Tabelle: \(u_a(t) = 1- \left(1-\frac{T_1}{T_2}\right)\cdot e^{-\frac{t}{T_2}}\)

\begin{sphinxuseclass}{cell}
\begin{sphinxuseclass}{tag_hide-input}\begin{sphinxVerbatimOutput}

\begin{sphinxuseclass}{cell_output}
\begin{sphinxVerbatim}[commandchars=\\\{\}]
[\PYGZlt{}matplotlib.axis.YTick at 0x2b4bcee9790\PYGZgt{},
 \PYGZlt{}matplotlib.axis.YTick at 0x2b4bcfe2810\PYGZgt{},
 \PYGZlt{}matplotlib.axis.YTick at 0x2b4bceb64d0\PYGZgt{},
 \PYGZlt{}matplotlib.axis.YTick at 0x2b4bcfa1590\PYGZgt{},
 \PYGZlt{}matplotlib.axis.YTick at 0x2b4bcf46b90\PYGZgt{},
 \PYGZlt{}matplotlib.axis.YTick at 0x2b4bcf9cd90\PYGZgt{},
 \PYGZlt{}matplotlib.axis.YTick at 0x2b4bcf9d6d0\PYGZgt{},
 \PYGZlt{}matplotlib.axis.YTick at 0x2b4bcf91bd0\PYGZgt{}]
\end{sphinxVerbatim}

\noindent\sphinxincludegraphics{{9ad376954878ce914a8e2c88fcbcfcea8e7da55fce9a09021c59507af3a70246}.png}

\end{sphinxuseclass}\end{sphinxVerbatimOutput}

\end{sphinxuseclass}
\end{sphinxuseclass}
\sphinxAtStartPar
Startwert ist \(k\cdot\frac{T_1}{T_2}\)\\
Endwert ist \(k\)

\begin{DUlineblock}{0em}
\item[] \sphinxstylestrong{\large OPV\sphinxhyphen{}Schaltung aus Blockdiagramm}
\end{DUlineblock}

\sphinxAtStartPar
Zum Rekonstruieren müssen bei einer Änderung jeweilige Elemente (je nach Veränderung; meist PT1 oder PD) dazugeschaltet werden,
bei den zusätzlichen Elementen muss die Verstärkung dimensioniert werden (bei allen außer dem ersten meist \(1\)),
und die Kreisfrequenz des Knicks dimensioniert werden.
Am Anfang muss das jeweilige Element ausgewählt werden welches die Funktion vor jedem Knick gut beschreibt

\sphinxAtStartPar
Um auf die gesamt OPV\sphinxhyphen{}Schaltung zu kommen müssen die OPV\sphinxhyphen{}Schaltungen der einzelnen Elemente aneinander gehängt werden (Vorzeichen beachten),
Die OPV Schaltungen sind alle Modifikationen des Invertierenden Verstärkers,
bei dem ein Widerstand mit einer Kapazität und Widerstand ausgetauscht,
je nach Element können diese in Serie oder Parallel liegen.
Für die Verschaltung siehe oben OPV\sphinxhyphen{}Schaltung

\begin{DUlineblock}{0em}
\item[] \sphinxstylestrong{\large OPV\sphinxhyphen{}Schaltung für Summen\sphinxhyphen{} und Differenzknoten}
\end{DUlineblock}

\sphinxAtStartPar
Summierknoten werden über OPV\sphinxhyphen{}Summierer/Subtrahierer realisiert

\sphinxAtStartPar
In den Schaltungen gilt:
\begin{itemize}
\item {} 
\sphinxAtStartPar
Ausgang: \(U_a\)

\item {} 
\sphinxAtStartPar
Eingänge je nach Vorzeichen
Evtl. Vorzeichen durch Invertierer anpassen

\end{itemize}

\begin{DUlineblock}{0em}
\item[] \sphinxstylestrong{\large OPV Subtrahierer}
\end{DUlineblock}

\sphinxAtStartPar
\(U_a = U_{e1} - U_{e2}\)

\begin{sphinxuseclass}{cell}
\begin{sphinxuseclass}{tag_hide-input}\begin{sphinxVerbatimOutput}

\begin{sphinxuseclass}{cell_output}
\noindent\sphinxincludegraphics{{1a8ec36ff771e80e6b23d3aba32effaf66c9eb21164c5e9cf1eb8f07f633bec2}.png}

\end{sphinxuseclass}\end{sphinxVerbatimOutput}

\end{sphinxuseclass}
\end{sphinxuseclass}
\begin{DUlineblock}{0em}
\item[] \sphinxstylestrong{\large inv OPV Summierer}
\end{DUlineblock}

\sphinxAtStartPar
\(U_a = -(U_{e1} + U_{e2})\)

\sphinxAtStartPar
für N\sphinxhyphen{}Eingänge gilt:
\(U_a = - \Sigma_{i=0}^{N}~U_{ei}\)

\begin{sphinxuseclass}{cell}
\begin{sphinxuseclass}{tag_hide-input}\begin{sphinxVerbatimOutput}

\begin{sphinxuseclass}{cell_output}
\noindent\sphinxincludegraphics{{0087eea71ec5c7f8c30badb77ef3e7ed183a211b6cf83cf2e15602bad55d1949}.png}

\end{sphinxuseclass}\end{sphinxVerbatimOutput}

\end{sphinxuseclass}
\end{sphinxuseclass}
\begin{DUlineblock}{0em}
\item[] \sphinxstylestrong{PT \(_2\) Element}
\end{DUlineblock}

\sphinxAtStartPar
Zusammen aus 1x IT1\sphinxhyphen{}Element in mit Rückkopplung (Schwingungsfähig)

\sphinxAtStartPar
Zwei PT1\sphinxhyphen{}Element (nicht Schwingungsfähig)

\begin{DUlineblock}{0em}
\item[] \sphinxstylestrong{\large Beschreibung im Frequenz\sphinxhyphen{} und Zeitbereich}
\end{DUlineblock}

\sphinxAtStartPar
\(G(s) = \frac{1}{1 + \frac{2D}{\omega_n} + s^2\frac{1}{\omega_n^2}}\)

\begin{DUlineblock}{0em}
\item[] \sphinxstylestrong{\large Kenngrößen}
\end{DUlineblock}
\begin{itemize}
\item {} 
\sphinxAtStartPar
Überschwingen: Amplitude der ersten Schwingung, wie hoch kommt das Signal überhaupt?
\(ü = \)

\item {} 
\sphinxAtStartPar
Verstärkung: Auf welchen Wert schwingt sich das Signal ein durch die Eingangsspannung

\item {} 
\sphinxAtStartPar
\(\Tau\): Exponentialkurve über die Amplituden legen, wann erreicht diese Kurve \(63%\) des Eingeschwungenen Zustandes (\(e^{-1}~\%\) vom Eingeschwungenen Zustand weg)

\item {} 
\sphinxAtStartPar
T: Periodendauer der Schwingung, mehrere Perioden messen und herunter zuteilen

\item {} 
\sphinxAtStartPar
\(T_ü\): Zeit bis zum Überschwingungsmaxima (keine ganze Periode bei starker Dämpfung) \(T_ü = \frac{T_0}{2}\)

\end{itemize}

\begin{DUlineblock}{0em}
\item[] \sphinxstylestrong{\large Überschwingen}
\end{DUlineblock}

\sphinxAtStartPar
Wenn sich das Signal auf \(1V\) einschwingt und bei der ersten Schwingung auf \(1.5V\) raufkommt, so ist \(ü=\frac{u_{max}}{k\cdot U_e}\)

\sphinxAtStartPar
Überschwingen ist manchmal gewollt muss jedoch auf die Situation angemessen dimensioniert werden.

\begin{DUlineblock}{0em}
\item[] \sphinxstylestrong{\large Schwingunsmaxima}
\end{DUlineblock}

\sphinxAtStartPar
Alternative Formel für \(ü\)\\
\(ü = e^{-\frac{\pi\cdot D}{\sqrt{1-D^2}}}\)

\begin{DUlineblock}{0em}
\item[] \sphinxstylestrong{\large Identifikation im Zeitbereich}
\end{DUlineblock}

\sphinxAtStartPar
\sphinxincludegraphics{{PT2-Bsp1}.png}

\sphinxAtStartPar
\sphinxincludegraphics{{PT2-Bsp2}.png}

\begin{DUlineblock}{0em}
\item[] \sphinxstylestrong{\large Bedeutung für die Regelungstechnik}
\end{DUlineblock}
\begin{itemize}
\item {} 
\sphinxAtStartPar
Wichtigste

\item {} 
\sphinxAtStartPar
Schwingungsfähig

\item {} 
\sphinxAtStartPar
Alles in Richtung PT2

\item {} 
\sphinxAtStartPar
Gut Beschrieben

\item {} 
\sphinxAtStartPar
Viele Faustregeln

\end{itemize}

\begin{DUlineblock}{0em}
\item[] \sphinxstylestrong{\large Nyquist Kriteritum}
\end{DUlineblock}

\begin{DUlineblock}{0em}
\item[] \sphinxstylestrong{\large Stabilitätsgrenzen}
\end{DUlineblock}

\sphinxAtStartPar
Wenn die Kurve über die die Frequenz auf der Realen und Imaginären Achse gezeichnet wird, so muss die Kurve kleiner 1 sein wenn die Kurve die x\sphinxhyphen{}Achse auf der negativen Seite schneidet.

\begin{sphinxuseclass}{cell}
\begin{sphinxuseclass}{tag_hide-input}\begin{sphinxVerbatimOutput}

\begin{sphinxuseclass}{cell_output}
\begin{sphinxVerbatim}[commandchars=\\\{\}]
\PYGZsq{}\PYGZsq{}
\end{sphinxVerbatim}

\noindent\sphinxincludegraphics{{2a1f1f6b24a681799030c9a2571bff6e20f1b16a37f70d0b61d6076edd0aadc5}.png}

\end{sphinxuseclass}\end{sphinxVerbatimOutput}

\end{sphinxuseclass}
\end{sphinxuseclass}
\begin{DUlineblock}{0em}
\item[] \sphinxstylestrong{\large Offener und geschlossener Regelkreis}
\end{DUlineblock}

\sphinxAtStartPar
Für die Anwendung des Kriteritums braucht man \sphinxhyphen{} bei uns \sphinxhyphen{} den Offenen Regelkreis,
die Übertragungsfunktion ist dabei in der Form:\\
\(G(s) = \frac{F_O(s)}{1 + F_O(s)}\)

\begin{sphinxuseclass}{cell}
\begin{sphinxuseclass}{tag_hide-input}\begin{sphinxVerbatimOutput}

\begin{sphinxuseclass}{cell_output}
\noindent\sphinxincludegraphics{{0b4b896809531e0a6f22a77542b7757d4736ded55d19aa69eaa68b66c65ab546}.png}

\end{sphinxuseclass}\end{sphinxVerbatimOutput}

\end{sphinxuseclass}
\end{sphinxuseclass}
\sphinxAtStartPar
Elemente welche nicht Teil der Rückkopplung sind werden bei der Stabilitätsprüfung \sphinxstylestrong{nicht} mitberücksichtigt!!

\begin{DUlineblock}{0em}
\item[] \sphinxstylestrong{\large Phasenrand}
\end{DUlineblock}

\sphinxAtStartPar
Bei einer Verstärkung von \(1\) wie weit ist man von den 180° (\(\pi\)) noch weg
\begin{itemize}
\item {} 
\sphinxAtStartPar
\(|F_O(i\omega_D)| = 1\)

\item {} 
\sphinxAtStartPar
\(\alpha_R = arg(F_O(i\omega_D))\)

\end{itemize}

\begin{DUlineblock}{0em}
\item[] \sphinxstylestrong{\large Amplitudenrand}
\end{DUlineblock}

\sphinxAtStartPar
Bei einer Phasendrehung von \(\pi\), welche Abstand hat man zur Verstärkung von \(1\)
\begin{itemize}
\item {} 
\sphinxAtStartPar
\(arg(F_O(i\omega_r)) = \pi\)

\item {} 
\sphinxAtStartPar
\(A_R = \frac{1}{|F_O(i\omega_r)|}\)

\end{itemize}

\begin{DUlineblock}{0em}
\item[] \sphinxstylestrong{\large Faustregeln (Kommt nicht)}
\end{DUlineblock}

\sphinxAtStartPar
Wenn man \(30%\) Überschwingen Einstellen will braucht man einen Phasenrand von ca \(40°\)







\renewcommand{\indexname}{Index}
\printindex
\end{document}